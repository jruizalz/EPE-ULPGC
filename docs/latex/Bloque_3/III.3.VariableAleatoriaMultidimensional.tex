
\documentclass[11pt]{article}

    
    \usepackage[breakable]{tcolorbox}
    \tcbset{nobeforeafter} % prevents tcolorboxes being placing in paragraphs
    \usepackage{float}
    \floatplacement{figure}{H} % forces figures to be placed at the correct location
    
    \usepackage[T1]{fontenc}
    % Nicer default font (+ math font) than Computer Modern for most use cases
    \usepackage{mathpazo}

    % Basic figure setup, for now with no caption control since it's done
    % automatically by Pandoc (which extracts ![](path) syntax from Markdown).
    \usepackage{graphicx}
    % We will generate all images so they have a width \maxwidth. This means
    % that they will get their normal width if they fit onto the page, but
    % are scaled down if they would overflow the margins.
    \makeatletter
    \def\maxwidth{\ifdim\Gin@nat@width>\linewidth\linewidth
    \else\Gin@nat@width\fi}
    \makeatother
    \let\Oldincludegraphics\includegraphics
    % Set max figure width to be 80% of text width, for now hardcoded.
    \renewcommand{\includegraphics}[1]{\Oldincludegraphics[width=.8\maxwidth]{#1}}
    % Ensure that by default, figures have no caption (until we provide a
    % proper Figure object with a Caption API and a way to capture that
    % in the conversion process - todo).
    \usepackage{caption}
    \DeclareCaptionLabelFormat{nolabel}{}
    \captionsetup{labelformat=nolabel}

    \usepackage{adjustbox} % Used to constrain images to a maximum size 
    \usepackage{xcolor} % Allow colors to be defined
    \usepackage{enumerate} % Needed for markdown enumerations to work
    \usepackage{geometry} % Used to adjust the document margins
    \usepackage{amsmath} % Equations
    \usepackage{amssymb} % Equations
    \usepackage{textcomp} % defines textquotesingle
    % Hack from http://tex.stackexchange.com/a/47451/13684:
    \AtBeginDocument{%
        \def\PYZsq{\textquotesingle}% Upright quotes in Pygmentized code
    }
    \usepackage{upquote} % Upright quotes for verbatim code
    \usepackage{eurosym} % defines \euro
    \usepackage[mathletters]{ucs} % Extended unicode (utf-8) support
    \usepackage[utf8x]{inputenc} % Allow utf-8 characters in the tex document
    \usepackage{fancyvrb} % verbatim replacement that allows latex
    \usepackage{grffile} % extends the file name processing of package graphics 
                         % to support a larger range 
    % The hyperref package gives us a pdf with properly built
    % internal navigation ('pdf bookmarks' for the table of contents,
    % internal cross-reference links, web links for URLs, etc.)
    \usepackage{hyperref}
    \usepackage{longtable} % longtable support required by pandoc >1.10
    \usepackage{booktabs}  % table support for pandoc > 1.12.2
    \usepackage[inline]{enumitem} % IRkernel/repr support (it uses the enumerate* environment)
    \usepackage[normalem]{ulem} % ulem is needed to support strikethroughs (\sout)
                                % normalem makes italics be italics, not underlines
    \usepackage{mathrsfs}
    

    
    % Colors for the hyperref package
    \definecolor{urlcolor}{rgb}{0,.145,.698}
    \definecolor{linkcolor}{rgb}{.71,0.21,0.01}
    \definecolor{citecolor}{rgb}{.12,.54,.11}

    % ANSI colors
    \definecolor{ansi-black}{HTML}{3E424D}
    \definecolor{ansi-black-intense}{HTML}{282C36}
    \definecolor{ansi-red}{HTML}{E75C58}
    \definecolor{ansi-red-intense}{HTML}{B22B31}
    \definecolor{ansi-green}{HTML}{00A250}
    \definecolor{ansi-green-intense}{HTML}{007427}
    \definecolor{ansi-yellow}{HTML}{DDB62B}
    \definecolor{ansi-yellow-intense}{HTML}{B27D12}
    \definecolor{ansi-blue}{HTML}{208FFB}
    \definecolor{ansi-blue-intense}{HTML}{0065CA}
    \definecolor{ansi-magenta}{HTML}{D160C4}
    \definecolor{ansi-magenta-intense}{HTML}{A03196}
    \definecolor{ansi-cyan}{HTML}{60C6C8}
    \definecolor{ansi-cyan-intense}{HTML}{258F8F}
    \definecolor{ansi-white}{HTML}{C5C1B4}
    \definecolor{ansi-white-intense}{HTML}{A1A6B2}
    \definecolor{ansi-default-inverse-fg}{HTML}{FFFFFF}
    \definecolor{ansi-default-inverse-bg}{HTML}{000000}

    % commands and environments needed by pandoc snippets
    % extracted from the output of `pandoc -s`
    \providecommand{\tightlist}{%
      \setlength{\itemsep}{0pt}\setlength{\parskip}{0pt}}
    \DefineVerbatimEnvironment{Highlighting}{Verbatim}{commandchars=\\\{\}}
    % Add ',fontsize=\small' for more characters per line
    \newenvironment{Shaded}{}{}
    \newcommand{\KeywordTok}[1]{\textcolor[rgb]{0.00,0.44,0.13}{\textbf{{#1}}}}
    \newcommand{\DataTypeTok}[1]{\textcolor[rgb]{0.56,0.13,0.00}{{#1}}}
    \newcommand{\DecValTok}[1]{\textcolor[rgb]{0.25,0.63,0.44}{{#1}}}
    \newcommand{\BaseNTok}[1]{\textcolor[rgb]{0.25,0.63,0.44}{{#1}}}
    \newcommand{\FloatTok}[1]{\textcolor[rgb]{0.25,0.63,0.44}{{#1}}}
    \newcommand{\CharTok}[1]{\textcolor[rgb]{0.25,0.44,0.63}{{#1}}}
    \newcommand{\StringTok}[1]{\textcolor[rgb]{0.25,0.44,0.63}{{#1}}}
    \newcommand{\CommentTok}[1]{\textcolor[rgb]{0.38,0.63,0.69}{\textit{{#1}}}}
    \newcommand{\OtherTok}[1]{\textcolor[rgb]{0.00,0.44,0.13}{{#1}}}
    \newcommand{\AlertTok}[1]{\textcolor[rgb]{1.00,0.00,0.00}{\textbf{{#1}}}}
    \newcommand{\FunctionTok}[1]{\textcolor[rgb]{0.02,0.16,0.49}{{#1}}}
    \newcommand{\RegionMarkerTok}[1]{{#1}}
    \newcommand{\ErrorTok}[1]{\textcolor[rgb]{1.00,0.00,0.00}{\textbf{{#1}}}}
    \newcommand{\NormalTok}[1]{{#1}}
    
    % Additional commands for more recent versions of Pandoc
    \newcommand{\ConstantTok}[1]{\textcolor[rgb]{0.53,0.00,0.00}{{#1}}}
    \newcommand{\SpecialCharTok}[1]{\textcolor[rgb]{0.25,0.44,0.63}{{#1}}}
    \newcommand{\VerbatimStringTok}[1]{\textcolor[rgb]{0.25,0.44,0.63}{{#1}}}
    \newcommand{\SpecialStringTok}[1]{\textcolor[rgb]{0.73,0.40,0.53}{{#1}}}
    \newcommand{\ImportTok}[1]{{#1}}
    \newcommand{\DocumentationTok}[1]{\textcolor[rgb]{0.73,0.13,0.13}{\textit{{#1}}}}
    \newcommand{\AnnotationTok}[1]{\textcolor[rgb]{0.38,0.63,0.69}{\textbf{\textit{{#1}}}}}
    \newcommand{\CommentVarTok}[1]{\textcolor[rgb]{0.38,0.63,0.69}{\textbf{\textit{{#1}}}}}
    \newcommand{\VariableTok}[1]{\textcolor[rgb]{0.10,0.09,0.49}{{#1}}}
    \newcommand{\ControlFlowTok}[1]{\textcolor[rgb]{0.00,0.44,0.13}{\textbf{{#1}}}}
    \newcommand{\OperatorTok}[1]{\textcolor[rgb]{0.40,0.40,0.40}{{#1}}}
    \newcommand{\BuiltInTok}[1]{{#1}}
    \newcommand{\ExtensionTok}[1]{{#1}}
    \newcommand{\PreprocessorTok}[1]{\textcolor[rgb]{0.74,0.48,0.00}{{#1}}}
    \newcommand{\AttributeTok}[1]{\textcolor[rgb]{0.49,0.56,0.16}{{#1}}}
    \newcommand{\InformationTok}[1]{\textcolor[rgb]{0.38,0.63,0.69}{\textbf{\textit{{#1}}}}}
    \newcommand{\WarningTok}[1]{\textcolor[rgb]{0.38,0.63,0.69}{\textbf{\textit{{#1}}}}}
    
    
    % Define a nice break command that doesn't care if a line doesn't already
    % exist.
    \def\br{\hspace*{\fill} \\* }
    % Math Jax compatibility definitions
    \def\gt{>}
    \def\lt{<}
    \let\Oldtex\TeX
    \let\Oldlatex\LaTeX
    \renewcommand{\TeX}{\textrm{\Oldtex}}
    \renewcommand{\LaTeX}{\textrm{\Oldlatex}}
    % Document parameters
    % Document title
    \title{III.3. Variable Aleatoria Multidimensional}
    
    
    
    
    
% Pygments definitions
\makeatletter
\def\PY@reset{\let\PY@it=\relax \let\PY@bf=\relax%
    \let\PY@ul=\relax \let\PY@tc=\relax%
    \let\PY@bc=\relax \let\PY@ff=\relax}
\def\PY@tok#1{\csname PY@tok@#1\endcsname}
\def\PY@toks#1+{\ifx\relax#1\empty\else%
    \PY@tok{#1}\expandafter\PY@toks\fi}
\def\PY@do#1{\PY@bc{\PY@tc{\PY@ul{%
    \PY@it{\PY@bf{\PY@ff{#1}}}}}}}
\def\PY#1#2{\PY@reset\PY@toks#1+\relax+\PY@do{#2}}

\expandafter\def\csname PY@tok@w\endcsname{\def\PY@tc##1{\textcolor[rgb]{0.73,0.73,0.73}{##1}}}
\expandafter\def\csname PY@tok@c\endcsname{\let\PY@it=\textit\def\PY@tc##1{\textcolor[rgb]{0.25,0.50,0.50}{##1}}}
\expandafter\def\csname PY@tok@cp\endcsname{\def\PY@tc##1{\textcolor[rgb]{0.74,0.48,0.00}{##1}}}
\expandafter\def\csname PY@tok@k\endcsname{\let\PY@bf=\textbf\def\PY@tc##1{\textcolor[rgb]{0.00,0.50,0.00}{##1}}}
\expandafter\def\csname PY@tok@kp\endcsname{\def\PY@tc##1{\textcolor[rgb]{0.00,0.50,0.00}{##1}}}
\expandafter\def\csname PY@tok@kt\endcsname{\def\PY@tc##1{\textcolor[rgb]{0.69,0.00,0.25}{##1}}}
\expandafter\def\csname PY@tok@o\endcsname{\def\PY@tc##1{\textcolor[rgb]{0.40,0.40,0.40}{##1}}}
\expandafter\def\csname PY@tok@ow\endcsname{\let\PY@bf=\textbf\def\PY@tc##1{\textcolor[rgb]{0.67,0.13,1.00}{##1}}}
\expandafter\def\csname PY@tok@nb\endcsname{\def\PY@tc##1{\textcolor[rgb]{0.00,0.50,0.00}{##1}}}
\expandafter\def\csname PY@tok@nf\endcsname{\def\PY@tc##1{\textcolor[rgb]{0.00,0.00,1.00}{##1}}}
\expandafter\def\csname PY@tok@nc\endcsname{\let\PY@bf=\textbf\def\PY@tc##1{\textcolor[rgb]{0.00,0.00,1.00}{##1}}}
\expandafter\def\csname PY@tok@nn\endcsname{\let\PY@bf=\textbf\def\PY@tc##1{\textcolor[rgb]{0.00,0.00,1.00}{##1}}}
\expandafter\def\csname PY@tok@ne\endcsname{\let\PY@bf=\textbf\def\PY@tc##1{\textcolor[rgb]{0.82,0.25,0.23}{##1}}}
\expandafter\def\csname PY@tok@nv\endcsname{\def\PY@tc##1{\textcolor[rgb]{0.10,0.09,0.49}{##1}}}
\expandafter\def\csname PY@tok@no\endcsname{\def\PY@tc##1{\textcolor[rgb]{0.53,0.00,0.00}{##1}}}
\expandafter\def\csname PY@tok@nl\endcsname{\def\PY@tc##1{\textcolor[rgb]{0.63,0.63,0.00}{##1}}}
\expandafter\def\csname PY@tok@ni\endcsname{\let\PY@bf=\textbf\def\PY@tc##1{\textcolor[rgb]{0.60,0.60,0.60}{##1}}}
\expandafter\def\csname PY@tok@na\endcsname{\def\PY@tc##1{\textcolor[rgb]{0.49,0.56,0.16}{##1}}}
\expandafter\def\csname PY@tok@nt\endcsname{\let\PY@bf=\textbf\def\PY@tc##1{\textcolor[rgb]{0.00,0.50,0.00}{##1}}}
\expandafter\def\csname PY@tok@nd\endcsname{\def\PY@tc##1{\textcolor[rgb]{0.67,0.13,1.00}{##1}}}
\expandafter\def\csname PY@tok@s\endcsname{\def\PY@tc##1{\textcolor[rgb]{0.73,0.13,0.13}{##1}}}
\expandafter\def\csname PY@tok@sd\endcsname{\let\PY@it=\textit\def\PY@tc##1{\textcolor[rgb]{0.73,0.13,0.13}{##1}}}
\expandafter\def\csname PY@tok@si\endcsname{\let\PY@bf=\textbf\def\PY@tc##1{\textcolor[rgb]{0.73,0.40,0.53}{##1}}}
\expandafter\def\csname PY@tok@se\endcsname{\let\PY@bf=\textbf\def\PY@tc##1{\textcolor[rgb]{0.73,0.40,0.13}{##1}}}
\expandafter\def\csname PY@tok@sr\endcsname{\def\PY@tc##1{\textcolor[rgb]{0.73,0.40,0.53}{##1}}}
\expandafter\def\csname PY@tok@ss\endcsname{\def\PY@tc##1{\textcolor[rgb]{0.10,0.09,0.49}{##1}}}
\expandafter\def\csname PY@tok@sx\endcsname{\def\PY@tc##1{\textcolor[rgb]{0.00,0.50,0.00}{##1}}}
\expandafter\def\csname PY@tok@m\endcsname{\def\PY@tc##1{\textcolor[rgb]{0.40,0.40,0.40}{##1}}}
\expandafter\def\csname PY@tok@gh\endcsname{\let\PY@bf=\textbf\def\PY@tc##1{\textcolor[rgb]{0.00,0.00,0.50}{##1}}}
\expandafter\def\csname PY@tok@gu\endcsname{\let\PY@bf=\textbf\def\PY@tc##1{\textcolor[rgb]{0.50,0.00,0.50}{##1}}}
\expandafter\def\csname PY@tok@gd\endcsname{\def\PY@tc##1{\textcolor[rgb]{0.63,0.00,0.00}{##1}}}
\expandafter\def\csname PY@tok@gi\endcsname{\def\PY@tc##1{\textcolor[rgb]{0.00,0.63,0.00}{##1}}}
\expandafter\def\csname PY@tok@gr\endcsname{\def\PY@tc##1{\textcolor[rgb]{1.00,0.00,0.00}{##1}}}
\expandafter\def\csname PY@tok@ge\endcsname{\let\PY@it=\textit}
\expandafter\def\csname PY@tok@gs\endcsname{\let\PY@bf=\textbf}
\expandafter\def\csname PY@tok@gp\endcsname{\let\PY@bf=\textbf\def\PY@tc##1{\textcolor[rgb]{0.00,0.00,0.50}{##1}}}
\expandafter\def\csname PY@tok@go\endcsname{\def\PY@tc##1{\textcolor[rgb]{0.53,0.53,0.53}{##1}}}
\expandafter\def\csname PY@tok@gt\endcsname{\def\PY@tc##1{\textcolor[rgb]{0.00,0.27,0.87}{##1}}}
\expandafter\def\csname PY@tok@err\endcsname{\def\PY@bc##1{\setlength{\fboxsep}{0pt}\fcolorbox[rgb]{1.00,0.00,0.00}{1,1,1}{\strut ##1}}}
\expandafter\def\csname PY@tok@kc\endcsname{\let\PY@bf=\textbf\def\PY@tc##1{\textcolor[rgb]{0.00,0.50,0.00}{##1}}}
\expandafter\def\csname PY@tok@kd\endcsname{\let\PY@bf=\textbf\def\PY@tc##1{\textcolor[rgb]{0.00,0.50,0.00}{##1}}}
\expandafter\def\csname PY@tok@kn\endcsname{\let\PY@bf=\textbf\def\PY@tc##1{\textcolor[rgb]{0.00,0.50,0.00}{##1}}}
\expandafter\def\csname PY@tok@kr\endcsname{\let\PY@bf=\textbf\def\PY@tc##1{\textcolor[rgb]{0.00,0.50,0.00}{##1}}}
\expandafter\def\csname PY@tok@bp\endcsname{\def\PY@tc##1{\textcolor[rgb]{0.00,0.50,0.00}{##1}}}
\expandafter\def\csname PY@tok@fm\endcsname{\def\PY@tc##1{\textcolor[rgb]{0.00,0.00,1.00}{##1}}}
\expandafter\def\csname PY@tok@vc\endcsname{\def\PY@tc##1{\textcolor[rgb]{0.10,0.09,0.49}{##1}}}
\expandafter\def\csname PY@tok@vg\endcsname{\def\PY@tc##1{\textcolor[rgb]{0.10,0.09,0.49}{##1}}}
\expandafter\def\csname PY@tok@vi\endcsname{\def\PY@tc##1{\textcolor[rgb]{0.10,0.09,0.49}{##1}}}
\expandafter\def\csname PY@tok@vm\endcsname{\def\PY@tc##1{\textcolor[rgb]{0.10,0.09,0.49}{##1}}}
\expandafter\def\csname PY@tok@sa\endcsname{\def\PY@tc##1{\textcolor[rgb]{0.73,0.13,0.13}{##1}}}
\expandafter\def\csname PY@tok@sb\endcsname{\def\PY@tc##1{\textcolor[rgb]{0.73,0.13,0.13}{##1}}}
\expandafter\def\csname PY@tok@sc\endcsname{\def\PY@tc##1{\textcolor[rgb]{0.73,0.13,0.13}{##1}}}
\expandafter\def\csname PY@tok@dl\endcsname{\def\PY@tc##1{\textcolor[rgb]{0.73,0.13,0.13}{##1}}}
\expandafter\def\csname PY@tok@s2\endcsname{\def\PY@tc##1{\textcolor[rgb]{0.73,0.13,0.13}{##1}}}
\expandafter\def\csname PY@tok@sh\endcsname{\def\PY@tc##1{\textcolor[rgb]{0.73,0.13,0.13}{##1}}}
\expandafter\def\csname PY@tok@s1\endcsname{\def\PY@tc##1{\textcolor[rgb]{0.73,0.13,0.13}{##1}}}
\expandafter\def\csname PY@tok@mb\endcsname{\def\PY@tc##1{\textcolor[rgb]{0.40,0.40,0.40}{##1}}}
\expandafter\def\csname PY@tok@mf\endcsname{\def\PY@tc##1{\textcolor[rgb]{0.40,0.40,0.40}{##1}}}
\expandafter\def\csname PY@tok@mh\endcsname{\def\PY@tc##1{\textcolor[rgb]{0.40,0.40,0.40}{##1}}}
\expandafter\def\csname PY@tok@mi\endcsname{\def\PY@tc##1{\textcolor[rgb]{0.40,0.40,0.40}{##1}}}
\expandafter\def\csname PY@tok@il\endcsname{\def\PY@tc##1{\textcolor[rgb]{0.40,0.40,0.40}{##1}}}
\expandafter\def\csname PY@tok@mo\endcsname{\def\PY@tc##1{\textcolor[rgb]{0.40,0.40,0.40}{##1}}}
\expandafter\def\csname PY@tok@ch\endcsname{\let\PY@it=\textit\def\PY@tc##1{\textcolor[rgb]{0.25,0.50,0.50}{##1}}}
\expandafter\def\csname PY@tok@cm\endcsname{\let\PY@it=\textit\def\PY@tc##1{\textcolor[rgb]{0.25,0.50,0.50}{##1}}}
\expandafter\def\csname PY@tok@cpf\endcsname{\let\PY@it=\textit\def\PY@tc##1{\textcolor[rgb]{0.25,0.50,0.50}{##1}}}
\expandafter\def\csname PY@tok@c1\endcsname{\let\PY@it=\textit\def\PY@tc##1{\textcolor[rgb]{0.25,0.50,0.50}{##1}}}
\expandafter\def\csname PY@tok@cs\endcsname{\let\PY@it=\textit\def\PY@tc##1{\textcolor[rgb]{0.25,0.50,0.50}{##1}}}

\def\PYZbs{\char`\\}
\def\PYZus{\char`\_}
\def\PYZob{\char`\{}
\def\PYZcb{\char`\}}
\def\PYZca{\char`\^}
\def\PYZam{\char`\&}
\def\PYZlt{\char`\<}
\def\PYZgt{\char`\>}
\def\PYZsh{\char`\#}
\def\PYZpc{\char`\%}
\def\PYZdl{\char`\$}
\def\PYZhy{\char`\-}
\def\PYZsq{\char`\'}
\def\PYZdq{\char`\"}
\def\PYZti{\char`\~}
% for compatibility with earlier versions
\def\PYZat{@}
\def\PYZlb{[}
\def\PYZrb{]}
\makeatother


    % For linebreaks inside Verbatim environment from package fancyvrb. 
    \makeatletter
        \newbox\Wrappedcontinuationbox 
        \newbox\Wrappedvisiblespacebox 
        \newcommand*\Wrappedvisiblespace {\textcolor{red}{\textvisiblespace}} 
        \newcommand*\Wrappedcontinuationsymbol {\textcolor{red}{\llap{\tiny$\m@th\hookrightarrow$}}} 
        \newcommand*\Wrappedcontinuationindent {3ex } 
        \newcommand*\Wrappedafterbreak {\kern\Wrappedcontinuationindent\copy\Wrappedcontinuationbox} 
        % Take advantage of the already applied Pygments mark-up to insert 
        % potential linebreaks for TeX processing. 
        %        {, <, #, %, $, ' and ": go to next line. 
        %        _, }, ^, &, >, - and ~: stay at end of broken line. 
        % Use of \textquotesingle for straight quote. 
        \newcommand*\Wrappedbreaksatspecials {% 
            \def\PYGZus{\discretionary{\char`\_}{\Wrappedafterbreak}{\char`\_}}% 
            \def\PYGZob{\discretionary{}{\Wrappedafterbreak\char`\{}{\char`\{}}% 
            \def\PYGZcb{\discretionary{\char`\}}{\Wrappedafterbreak}{\char`\}}}% 
            \def\PYGZca{\discretionary{\char`\^}{\Wrappedafterbreak}{\char`\^}}% 
            \def\PYGZam{\discretionary{\char`\&}{\Wrappedafterbreak}{\char`\&}}% 
            \def\PYGZlt{\discretionary{}{\Wrappedafterbreak\char`\<}{\char`\<}}% 
            \def\PYGZgt{\discretionary{\char`\>}{\Wrappedafterbreak}{\char`\>}}% 
            \def\PYGZsh{\discretionary{}{\Wrappedafterbreak\char`\#}{\char`\#}}% 
            \def\PYGZpc{\discretionary{}{\Wrappedafterbreak\char`\%}{\char`\%}}% 
            \def\PYGZdl{\discretionary{}{\Wrappedafterbreak\char`\$}{\char`\$}}% 
            \def\PYGZhy{\discretionary{\char`\-}{\Wrappedafterbreak}{\char`\-}}% 
            \def\PYGZsq{\discretionary{}{\Wrappedafterbreak\textquotesingle}{\textquotesingle}}% 
            \def\PYGZdq{\discretionary{}{\Wrappedafterbreak\char`\"}{\char`\"}}% 
            \def\PYGZti{\discretionary{\char`\~}{\Wrappedafterbreak}{\char`\~}}% 
        } 
        % Some characters . , ; ? ! / are not pygmentized. 
        % This macro makes them "active" and they will insert potential linebreaks 
        \newcommand*\Wrappedbreaksatpunct {% 
            \lccode`\~`\.\lowercase{\def~}{\discretionary{\hbox{\char`\.}}{\Wrappedafterbreak}{\hbox{\char`\.}}}% 
            \lccode`\~`\,\lowercase{\def~}{\discretionary{\hbox{\char`\,}}{\Wrappedafterbreak}{\hbox{\char`\,}}}% 
            \lccode`\~`\;\lowercase{\def~}{\discretionary{\hbox{\char`\;}}{\Wrappedafterbreak}{\hbox{\char`\;}}}% 
            \lccode`\~`\:\lowercase{\def~}{\discretionary{\hbox{\char`\:}}{\Wrappedafterbreak}{\hbox{\char`\:}}}% 
            \lccode`\~`\?\lowercase{\def~}{\discretionary{\hbox{\char`\?}}{\Wrappedafterbreak}{\hbox{\char`\?}}}% 
            \lccode`\~`\!\lowercase{\def~}{\discretionary{\hbox{\char`\!}}{\Wrappedafterbreak}{\hbox{\char`\!}}}% 
            \lccode`\~`\/\lowercase{\def~}{\discretionary{\hbox{\char`\/}}{\Wrappedafterbreak}{\hbox{\char`\/}}}% 
            \catcode`\.\active
            \catcode`\,\active 
            \catcode`\;\active
            \catcode`\:\active
            \catcode`\?\active
            \catcode`\!\active
            \catcode`\/\active 
            \lccode`\~`\~ 	
        }
    \makeatother

    \let\OriginalVerbatim=\Verbatim
    \makeatletter
    \renewcommand{\Verbatim}[1][1]{%
        %\parskip\z@skip
        \sbox\Wrappedcontinuationbox {\Wrappedcontinuationsymbol}%
        \sbox\Wrappedvisiblespacebox {\FV@SetupFont\Wrappedvisiblespace}%
        \def\FancyVerbFormatLine ##1{\hsize\linewidth
            \vtop{\raggedright\hyphenpenalty\z@\exhyphenpenalty\z@
                \doublehyphendemerits\z@\finalhyphendemerits\z@
                \strut ##1\strut}%
        }%
        % If the linebreak is at a space, the latter will be displayed as visible
        % space at end of first line, and a continuation symbol starts next line.
        % Stretch/shrink are however usually zero for typewriter font.
        \def\FV@Space {%
            \nobreak\hskip\z@ plus\fontdimen3\font minus\fontdimen4\font
            \discretionary{\copy\Wrappedvisiblespacebox}{\Wrappedafterbreak}
            {\kern\fontdimen2\font}%
        }%
        
        % Allow breaks at special characters using \PYG... macros.
        \Wrappedbreaksatspecials
        % Breaks at punctuation characters . , ; ? ! and / need catcode=\active 	
        \OriginalVerbatim[#1,codes*=\Wrappedbreaksatpunct]%
    }
    \makeatother

    % Exact colors from NB
    \definecolor{incolor}{HTML}{303F9F}
    \definecolor{outcolor}{HTML}{D84315}
    \definecolor{cellborder}{HTML}{CFCFCF}
    \definecolor{cellbackground}{HTML}{F7F7F7}
    
    % prompt
    \newcommand{\prompt}[4]{
        \llap{{\color{#2}[#3]: #4}}\vspace{-1.25em}
    }
    

    
    % Prevent overflowing lines due to hard-to-break entities
    \sloppy 
    % Setup hyperref package
    \hypersetup{
      breaklinks=true,  % so long urls are correctly broken across lines
      colorlinks=true,
      urlcolor=urlcolor,
      linkcolor=linkcolor,
      citecolor=citecolor,
      }
    % Slightly bigger margins than the latex defaults
    
    \geometry{verbose,tmargin=1in,bmargin=1in,lmargin=1in,rmargin=1in}
    
    

    \begin{document}
    
    
    \maketitle
    
    

    
    \hypertarget{variables-aleatorias-n-dimensionales}{%
\section{Variables Aleatorias
N-Dimensionales}\label{variables-aleatorias-n-dimensionales}}

Al introducir el concepto de variable aleatoria conjunta, ya se mencionó
que los vectores aleatorios pueden tener un número arbitrario \(M\) de
componentes:

\[\mathbf X = (X_1, \ldots X_M)^\mathrm{T}\]

Ya hemos visto en profundidad la caracterización conjunta y marginal
para el caso de dos variables aleatorias (vector aleatorio de dimesión
\(2\)). La extensión a \(M\) componentes es inmediata:

La \textbf{Función de Distribución Conjunta}
\(F_\mathbf{X}(\mathbf{x}) \equiv F_{X_1\ldots X_M}(x_1,\ldots x_M)\) es

\[
F_\mathbf{X}(\mathbf{x}) \equiv F_{X_1\ldots X_M}(x_1,\ldots x_M) = 
P(X_1 \leq x_1, \ldots X_M \leq x_M)
\]

Las \textbf{funciones de distribución marginales} se obtienen sin más
que evaluar en \(\infty\) las variables que se desean descartar. Por
ejemplo:

\[
F_{X_1X_3}(x_1, x_3) = F_{X_1\ldots X_N}(x_1,\infty, x_3, \infty , \ldots \infty)
\]

\[
F_{X_3}(x_3) = F_{X_1 \ldots X_N}(\infty,\infty, x_3, \infty , \ldots \infty)
\]

Por supuesto,
\(F_\mathbf{X}(\boldsymbol{\infty}) \equiv F_{X_1\ldots X_M}(\infty,\ldots \infty) = 1\)

    Suele resultar oportuno definir el vector aleatorio \(\mathbf{V}\) con
dos componentes que son, a su vez, vectores aleatorios,
\(\mathbf X = (X_1, \ldots X_M)^\mathrm{T}\) e
\(\mathbf X = (X_1, \ldots X_M)^\mathrm{T}\):

\[\mathbf V = (\mathbf{X}^\mathrm{T} , \mathbf{Y}^\mathrm{T})^\mathrm{T} = 
(X_1, \ldots X_M , Y_1 \ldots Y_M)^\mathrm{T}\]

Ello permite aplicar de forma prácticamente directa las expresiones
vistas para la caracterización del vector bidimensional \((X,Y)\) al
vector \(M+N\)-dimensional
\(\mathbf V = (\mathbf{X}^\mathrm{T} , \mathbf{Y}^\mathrm{T})^\mathrm{T}\),
recurriendo a ciertos \emph{abusos de notación} muy sencillos de
entender. Por ejemplo:

\[
F_\mathbf{V}(\mathbf{v}) \equiv F_{\mathbf{X}\mathbf{Y}}(\mathbf{x},\mathbf{y}) = 
P(X_1 \leq x_1, \ldots X_M \leq x_M, Y_1 \leq y_1 \ldots Y_M \leq y_M) \equiv 
P(\mathbf{X} \leq \mathbf{x}, \mathbf{Y} \leq \mathbf{y})
\]

\[
F_{\mathbf{X}}(\mathbf{x}) = F_{\mathbf{X}\mathbf{Y}}(\mathbf{x},\boldsymbol{\infty}) \equiv
F_{X_1\ldots X_M Y_1 \ldots Y_N}(x_1,\ldots x_M , \infty , \ldots \infty) = F_{X_1\ldots X_M}(x_1,\ldots x_M)
\]

\[
F_{\mathbf{Y}}(\mathbf{y}) = F_{\mathbf{X}\mathbf{Y}}(\mathbf{\boldsymbol{\infty}},\mathbf{y}) \equiv
F_{X_1\ldots X_M Y_1 \ldots Y_N}(\infty,\ldots \infty , y_1 , \ldots y_N) = F_{Y_1\ldots Y_N}(y_1,\ldots y_N)
\]

    \hypertarget{funciuxf3n-de-densidad-conjunta-multidimensional}{%
\subsubsection{Función de densidad conjunta
multidimensional}\label{funciuxf3n-de-densidad-conjunta-multidimensional}}

La \textbf{función de densidad conjunta} del vector aleatorio
\(\mathbf X\), suponiendo que está compuesto por \textbf{variables
aleatorias continuas}, es:

\[
f_\mathbf{X}(\mathbf{x}) \equiv f_{X_1\ldots X_M}(x_1,\ldots x_M) = \frac{\partial^{M} F_{X_1 \ldots X_M}\left(x_{1}, \ldots, x_{M}\right)}{\partial x_{1}, \ldots, \partial x_{M}} \equiv
\frac{\boldsymbol{\partial} F_\mathbf{X}(\mathbf{x})}{\boldsymbol{\partial} \mathbf{x}}
\]

Por tanto, puede obtenerse la función de distribución a partir de la de
densidad conjunta:

\[
F_\mathbf{X}(\mathbf{x}) \equiv F_{X_1\ldots X_M}(x_1,\ldots x_M) = \int_{-\infty}^{x_1} \ldots \int_{-\infty}^{x_M} f_{X_1\ldots X_M}(x_1,\ldots x_M) dx_1 \ldots dx_M \equiv 
\boldsymbol{\int}_{\boldsymbol{-\infty}}^{\mathbf{x}} f_\mathbf{X}(\mathbf{x})\mathbf{dx} 
\]

    \textbf{La masa total de probabilidad es uno}:

\[
\int_{-\infty}^{\infty} \ldots \int_{-\infty}^{\infty} f_{X_1\ldots X_M}(x_1,\ldots x_M) dx_1 \ldots dx_M = F_{X_1 \ldots X_M}(\infty,\ldots \infty) = 1 \equiv \\
\equiv \boldsymbol{\int}_{\boldsymbol{-\infty}}^{\boldsymbol{\infty}} f_\mathbf{X}(\mathbf{x})\mathbf{dx}
= F_\mathbf{X}(\boldsymbol{\infty}) = 1
\]

Las \textbf{funciones de densidad marginales} se obtienen sin más que
integrar las variables que se desean descartar. Por ejemplo:

\[
f_{X_1X_3}(x_1,x_3) = \int_{-\infty}^{\infty}\ldots \int_{-\infty}^{\infty} f_{X_1\ldots X_M} (x_1, \ldots x_M) dx_2 dx_4 \ldots dx_M
\]

\[
f_{X_1}(x_1) = \int_{-\infty}^{\infty}\ldots \int_{-\infty}^{\infty} f_{X_1\ldots X_M} (x_1, \ldots x_M) dx_2 \ldots dx_M
\]

    Considerando ahora el vector aleatorio \(\mathbf{V}\) definido como
antes:

\[
f_{\mathbf{V}}(\mathbf{v}) \equiv f_\mathbf{XY}(\mathbf{x},\mathbf{y}) = 
\frac{\boldsymbol{\partial}^2 F_{\mathbf{XY}}(\mathbf{x},\mathbf{y})}{\boldsymbol{\partial} \mathbf{x}\boldsymbol{\partial} \mathbf{y}}
\]

\[
F_\mathbf{V}(\mathbf{v}) \equiv F_{\mathbf{X} \mathbf{Y}}(\mathbf{x},\mathbf{y}) = \int_{-\boldsymbol{\infty}}^{\mathbf{x}}\int_{-\boldsymbol{\infty}}^{\mathbf{y}} f_{\mathbf{X}\mathbf{Y}}(\mathbf{x},\mathbf{y}) \mathbf{dx dy} \equiv 
\boldsymbol{\int}_{\boldsymbol{-\infty}}^{\mathbf{v}} f_\mathbf{V}(\mathbf{v})\mathbf{dv} 
\]

Y las funciones de densidad marginales de ambos vectores componentes
son:

\[
f_{\mathbf{X}}(\mathbf{x}) = \boldsymbol{\int}_{-\boldsymbol{\infty}}^{\boldsymbol{\infty}} f_{\mathbf{XY}}(\mathbf{x}, \mathbf{y}) \mathbf{dy}
\]

\[
f_{\mathbf{Y}}(\mathbf{y}) = \boldsymbol{\int}_{-\boldsymbol{\infty}}^{\boldsymbol{\infty}} f_{\mathbf{XY}}(\mathbf{x}, \mathbf{y}) \mathbf{dx}
\]

Por supuesto, \(\int_{-\boldsymbol{\infty}}^{\boldsymbol{\infty}}\int_{-\boldsymbol{\infty}}^{\boldsymbol{\infty}} f_{\mathbf{X}\mathbf{Y}}(\mathbf{x},\mathbf{y}) \mathbf{dx dy} = F_{\mathbf{X} \mathbf{Y}}(\infty,\infty) = 1\)


    Como ya sabemos, las funciones marginales no permiten conocer la
conjunta salvo que sean independientes. La \textbf{independencia
conjunta} requiere que todas las combinaciones posibles de variables
aleatorias componentes sean independientes. En tal caso:

\[
F_{X_1 \ldots X_M}(x_1 \ldots x_M) = F_{X_1}(x_1) \ldots F_{X_M}(x_M)
\]

\[
f_{X_1 \ldots X_M}(x_1 \ldots x_M) = f_{X_1}(x_1) \ldots f_{X_M}(x_M)
\]

En el caso de que tengamos dos vectores aleatorios \(\mathbf{X}\),
\(\mathbf{Y}\) independientes entre sí (pero no necesariamente las
componentes de cada uno de ellos por separado):

\[
F_{\mathbf{XY}}(\mathbf{x},\mathbf{y}) = F_{\mathbf{X}}(\mathbf{x}) F_{\mathbf{Y}}(\mathbf{y})
\]

\[
f_{\mathbf{XY}}(\mathbf{x},\mathbf{y}) = f_{\mathbf{X}}(\mathbf{x}) f_{\mathbf{Y}}(\mathbf{y})
\]

    \hypertarget{funciuxf3n-de-masa-de-probabilidad-conjunta}{%
\subsubsection{Función de masa de probabilidad
conjunta}\label{funciuxf3n-de-masa-de-probabilidad-conjunta}}

Si las componentes del vector aleatorio son \textbf{variables aleatorias
discretas}, la función de distribución resulta discontinua y no es
derivable. Como ya hemos visto, en tal caso recurrimos a la
\textbf{función de masa de probabilidad conjunta}

\[
p_{\mathbf{X}}(\mathbf{x_i}) \equiv p_{X_1 \ldots X_M}(x_{i_1}, \ldots x_{i_M}) = P(X_1 = x_{i_1} , \ldots X_M = x_{i_M})
\]

\[
p_{\mathbf{V}}(\mathbf{v_{k}}) \equiv p_{\mathbf{X} \mathbf{Y}}(\mathbf{x_{i}}, \mathbf{y_{j}}) = P(\mathbf{X} = \mathbf{x_{i}} , \mathbf{Y} = \mathbf{y_{j}})
\]

donde \(\mathbf{i} =(i_1 \ldots i_M)\), \(\mathbf{j} =(j_1 \ldots j_N)\)
y \(\mathbf{k}\) es la concatenación de ambos. La \textbf{función de
distribución conjunta} se obtiene acumulando la función de masa de
probabilidad conjunta:

\[
F_\mathbf{X}(\mathbf{x_i}) \equiv F_{X_1\ldots X_M}(x_{i_1},\ldots x_{i_M}) = \sum_{p_1=-\infty}^{i_1} \ldots \sum_{p_M=-\infty}^{i_M} f_{X_1\ldots X_M}(x_{p_1},\ldots x_{p_M}) \equiv 
\boldsymbol{\sum}_{\mathbf{p}=\boldsymbol{-\infty}}^{\mathbf{i}} f_\mathbf{X}(\mathbf{x_p}) 
\]

\[
F_\mathbf{V}(\mathbf{v_k}) \equiv F_{\mathbf{X} \mathbf{Y}}(\mathbf{x_i},\mathbf{y_j}) = \sum_{\mathbf{p}=-\boldsymbol{\infty}}^{\mathbf{i}}\sum_{\mathbf{q}=-\boldsymbol{\infty}}^{\mathbf{j}} p_{\mathbf{X}\mathbf{Y}}(\mathbf{x_p},\mathbf{y_q}) \equiv 
\boldsymbol{\sum}_{\mathbf{r}=\boldsymbol{-\infty}}^{\mathbf{k}} f_\mathbf{V}(\mathbf{v_r}) 
\]

    \hypertarget{momentos-conjuntos-vector-de-medias-y-matrices-de-covarianza}{%
\subsubsection{Momentos conjuntos: Vector de medias y matrices de
covarianza}\label{momentos-conjuntos-vector-de-medias-y-matrices-de-covarianza}}

El \textbf{vector de medias} se obtiene a partir de la esperanza
matemática de las funciones marginales de cada una de las variables
aleatorias, como ya se vio en el caso bidimensional:

\[
E(\mathbf{X})\equiv \boldsymbol{\mu}_{\mathbf{X}} = \left(E(X_1), \ldots E(X_M) \right)^\mathrm{T}
\]

\[
E(\mathbf{V})\equiv \boldsymbol{\mu}_{\mathbf{V}} 
= \left(\boldsymbol{\mu}_\mathbf{X}^\mathrm{T} , \boldsymbol{\mu}_\mathbf{Y}^\mathrm{T} \right)^\mathrm{T}
= \left(E(\mathbf{X})^\mathrm{T}, E(\mathbf{Y})^\mathrm{T} \right)^\mathrm{T} =
\left(E(X_1), \ldots E(X_M), E(Y_1), \ldots E(Y_N)\right)^\mathrm{T}
\]

La \textbf{matriz de autocorrelación} del vector aleatorio
\(\mathbf{X}\) es:

\[
\mathbf{R_{XX}} = E(\mathbf{XX}^\mathrm{T}) = 
\begin{pmatrix}
E(X_1^2) & \ldots & R_{X_1X_M}\\
\vdots & \ddots & \vdots\\
R_{X_MX_1} & \ldots & E(X_M^2)
\end{pmatrix}
\]

que resulta \textbf{simétrica y semidefinida positiva}. Si todas las
componentes son ortogonales entre sí, la matriz resulta diagonal.

    La \textbf{matriz de correlación cruzada} de los vectores aleatorios
\(X\) e \(Y\) es:

\[
\mathbf{R_{XY}} = E(\mathbf{XY}^\mathrm{T}) = 
\begin{pmatrix}
R_{X_1Y_1} & \ldots & R_{X_1Y_N}\\
\vdots & \ddots & \vdots\\
R_{X_MY_1} & \ldots & R_{X_MY_N}
\end{pmatrix}
\]

y se cumple que \(\mathbf{R_{YX}} = \mathbf{R_{XY}}^\mathrm{T}\). Por
tanto, la matriz de autocorrelación del vector
\(\mathbf V = (\mathbf{X}^\mathrm{T} , \mathbf{Y}^\mathrm{T})^\mathrm{T}\)
es

\[
\mathbf{R_{V}} = E(\mathbf{VV}^\mathrm{T}) = 
\begin{pmatrix}
\mathbf{R_{XX}} & \mathbf{R_{XY}}\\
\mathbf{R_{YX}} & \mathbf{R_{YY}}
\end{pmatrix}
\]

que también resulta simétrica y semidefinida positiva.

    La \textbf{matriz de autocovarianza} del vector aleatorio \(\mathbf{X}\)
es:

\[
\mathbf{C_{XX}} = E((\mathbf{X}-\mathbf{\mu_X})(\mathbf{X}-\mathbf{\mu_X})^\mathrm{T}) = 
\begin{pmatrix}
\sigma_{X_1}^2 & \ldots & C_{X_1X_M}\\
\vdots & \ddots & \vdots\\
C_{X_MX_1} & \ldots & \sigma_{X_M}^2
\end{pmatrix} =
\mathbf{R_{XX}}-\boldsymbol{\mu}_{\mathbf X}\boldsymbol{\mu}_{\mathbf X}^\mathrm{T}
\]

que resulta \textbf{simétrica y semidefinida positiva}. Si todas las
componentes son \textbf{incorreladas} entre sí, la matriz resulta
\textbf{diagonal}:
\(\mathbf{C_{XX}} = \mathrm{\mathbf{diag}}[\sigma_{X_1}^2\ldots \sigma_{X_M}^2]\)

Las covarianzas de las componentes (elementos fuera de la diagonal)
pueden ponerse en función de los coeficientes de correlación:

\[
C_{X_iX_j} = \rho_{X_iX_j}\sigma_{X_i}\sigma_{X_j}
\]

    La \textbf{matriz de covarianza cruzada} de los vectores aleatorios
\(X\) e \(Y\) es:

\[
\mathbf{C_{XY}} = E((\mathbf{X}-\mathbf{\mu_X})(\mathbf{Y}-\mathbf{\mu_Y})^\mathrm{T}) = 
\begin{pmatrix}
C_{X_1Y_1} & \ldots & C_{X_1Y_N}\\
\vdots & \ddots & \vdots\\
C_{X_MY_1} & \ldots &  C_{X_MY_N}
\end{pmatrix} =
\mathbf{R_{XX}}-\boldsymbol{\mu}_{\mathbf X}\boldsymbol{\mu}_{\mathbf Y}^\mathrm{T}
\]

y se cumple que \(\mathbf{C_{YX}} = \mathbf{C_{XY}}^\mathrm{T}\). Por
tanto, la matriz de autocovarianza del vector
\(\mathbf V = (\mathbf{X}^\mathrm{T} , \mathbf{Y}^\mathrm{T})^\mathrm{T}\)
es

\[
\mathbf{C_{V}} = E((\mathbf{V}-\mathbf{\mu_V})(\mathbf{V}-\mathbf{\mu_V})^\mathrm{T}) =
\begin{pmatrix}
\mathbf{C_{XX}} & \mathbf{C_{XY}}\\
\mathbf{C_{YX}} & \mathbf{C_{YY}}
\end{pmatrix}
\]

que también resulta simétrica y semidefinida positiva.

    \hypertarget{distribuciones-condicionadas.-teoremas-de-bayes-y-de-la-probabilidad-total}{%
\subsubsection{Distribuciones condicionadas. Teoremas de Bayes y de la
Probabilidad
Total}\label{distribuciones-condicionadas.-teoremas-de-bayes-y-de-la-probabilidad-total}}

Consideremos el vector aleatorio
\(\mathbf{X} = (X_1, \ldots X_{p-1}, X_{p}, X_{p+1} \ldots X_M)^\mathrm{T}\).
Podemos obtener la \textbf{función de distribución condicionada} de unas
componentes por otras:

\[
F_{X_1 \ldots X_{p-1} | X_p \ldots X_M} (x_1 , \ldots x_{p-1} | x_p , \ldots x_M) = \frac{F_{X_1 \ldots X_M}(x_1 , \ldots x_M)}{F_{X_p \ldots X_M}(x_p, \ldots x_M)}
\]

Considerando que las variables aleatorias son continuas, la
\textbf{función de densidad condicionada} es:

\[
f_{X_1 \ldots X_{p-1} | X_p \ldots X_M} (x_1 , \ldots x_{p-1} | x_p , \ldots x_M) = \frac{f_{X_1 \ldots X_M}(x_1 , \ldots x_M)}{f_{X_p \ldots X_M}(x_p, \ldots x_M)}
\]

Si las variables aleatorias son discretas, la \textbf{función de masa de
probabilidad} resulta:

\[
p_{X_1 \ldots X_{p-1} | X_p \ldots X_M} (x_{i_1} , \ldots x_{i_{p-1}} | x_{i_p} , \ldots x_{i_M}) = \frac{p_{X_1 \ldots X_M}(x_{i_1} , \ldots x_{i_M})}{p_{X_{p} \ldots X_M}(x_{i_p}, \ldots x_{i_M})}
\]

    \hypertarget{regla-de-la-cadena}{%
\paragraph{Regla de la cadena}\label{regla-de-la-cadena}}

Aplicando sucesiva y ordenadamente la definición de función de densidad
condicionada:

\begin{align}
f_{X_1 \ldots X_M}(x_1 \ldots x_M) 
&= 
f_{X_1 | X_2 \ldots X_M} (x_1 | x_2 \ldots x_{M}) \ldots f_{X_{M-1} | X_M} (x_{M-1} | x_M) f_{X_M} (x_M)\\
&= 
f_{X_M | X_{M-1} \ldots X_1} (x_M | x_{M-1} \ldots x_{1}) \ldots f_{X_2 | X_1} (x_2 | x_1) f_{X_1} (x_1)
\end{align}

Adviértase que no hay un orden establecido por el cual haya que
condicionar las funciones de densidad. Además de estas dos formas de
construir la cadena de productos, podrían haberse elaborado otras
extrayendo en distinto orden las variables condicionantes. Cuando veamos
secuencias o procesos temporales, la causalidad establecerá un
ordenamiento que sí habrá que respetar.

Si las variables son discretas, puede obtenerse una descomposición
idéntica utilizando funciones de masa de probabilidad.

    \hypertarget{ecuaciuxf3n-de-chapman-kolmogorov}{%
\paragraph{Ecuación de
Chapman-Kolmogorov}\label{ecuaciuxf3n-de-chapman-kolmogorov}}

Se obtiene combinando la definición de función de densidad o de masa de
probbailidad con la obtención de funciones marginales. Supone una
\emph{extensión del Teorema de la Probabilidad Total}. Veámoslo con unos
ejemplos:

\[
f_{X_3 | X_1}(x_3 | x_1) = \int_{-\infty}^{\infty} f_{X_3X_2 | X_1}(x_3, x_2 | x_1) dx_2 =
\int_{-\infty}^{\infty} f_{X_3 | X_2 X_1}(x_3 | x_2 , x_1) f_{X_2}(x_2 | x_1) dx_2
\]

\[
f_{X_1 | X_4}(x_1 | x_4) = \int_{-\infty}^{\infty}\int_{-\infty}^{\infty} f_{X_1 X_2 X_3 | X_4}(x_1 , x_2 , x_3| x_4) dx_2dx_3=\\
= \int_{-\infty}^{\infty}\int_{-\infty}^{\infty} f_{X_1 | X_2 X_3 X_4}(x_1 | x_2 , x_3 , x_4) f_{X_2X_3 | X_4}(x_2, x_3 | x_4) dx_2dx_3
\]

La ecuación puede aplicarse al caso discreto sin más que sustituir las
funciones de densidad por funciones de masa de probabilidad y las
integrales por sumatorios

    La \textbf{función de distribución de un vector aleatorio,
\(\mathbf{Y}\), condicionada por otro}, \(\mathbf{X}\), o viceversa, se
define extendiendo la definición ya vista para dos variables alatorias

\[
F_{\mathbf{Y}|\mathbf{X}}(\mathbf{y}|\mathbf{x}) = \frac{F_{\mathbf{XY}}(\mathbf{x},\mathbf{y})}{F_{\mathbf{X}}(\mathbf{x})} \qquad
F_{\mathbf{X}|\mathbf{Y}}(\mathbf{x}|\mathbf{y}) = \frac{F_{\mathbf{XY}}(\mathbf{x},\mathbf{y})}{F_{\mathbf{Y}}(\mathbf{y})}
\]

A partir de tal definición es inmediato obtener, para el \textbf{caso
continuo}, la \textbf{función de densidad condicionada}:

\[
f_{\mathbf{Y}|\mathbf{X}}(\mathbf{y}|\mathbf{x}) = \frac{f_{\mathbf{XY}}(\mathbf{x},\mathbf{y})}{f_{\mathbf{X}}(\mathbf{x})} \qquad
f_{\mathbf{X}|\mathbf{Y}}(\mathbf{x}|\mathbf{y}) = \frac{f_{\mathbf{XY}}(\mathbf{x},\mathbf{y})}{f_{\mathbf{Y}}(\mathbf{y})}
\]

Y, para el \textbf{caso discreto}, la \textbf{función de masa de
probabilidad condicionada}:

\[
p_{\mathbf{Y}|\mathbf{X}}(\mathbf{y_j}|\mathbf{x_i}) = \frac{p_{\mathbf{XY}}(\mathbf{x_i},\mathbf{y_j})}{p_{\mathbf{X}}(\mathbf{x_i})} \qquad
p_{\mathbf{X}|\mathbf{Y}}(\mathbf{x_i}|\mathbf{y_j}) = \frac{p_{\mathbf{XY}}(\mathbf{x_i},\mathbf{y_j})}{p_{\mathbf{Y}}(\mathbf{y_j})}
\]

    \hypertarget{transformaciones-lineales-y-teorema-del-luxedmite-central}{%
\subsubsection{Transformaciones Lineales y Teorema del Límite
Central}\label{transformaciones-lineales-y-teorema-del-luxedmite-central}}

Es posible definir transformaciones \(\mathbf{Y} = \mathbf{g(X)}\) que
convierten el vector aleatorio \(\mathbf{X}\) en otro \(\mathbf{Y}\),
cuya caracterización conjunta es posible obtener. Entre todas ellas,
vamos a prestarle atención a las \textbf{transformaciones lineales}. Por
ejemplo:

\[Y = k_1 X_1 + \ldots k_M X_M = \mathbf{k}^\mathrm{T}\mathbf{X}\]

La media resulta:

\[
\mu_Y \equiv E(Y) = \mathbf{k}^\mathrm{T}\boldsymbol{\mu}_\mathbf{X}
\]

Y la varianza:

\[
\sigma_{Y}^2 = E\left(\mathbf{k}^\mathrm{T}(\mathbf{X}-\boldsymbol{\mu}_\mathbf{X}) (\mathbf{X}-\boldsymbol{\mu}_\mathbf{X} )^\mathrm{T}\mathbf{k}
\right)
= \mathbf{k}^\mathrm{T}\mathbf{C_{XX}}\mathbf{k}
\]

Si el vector aleatorio \(\mathbf{X}\) tiene sus componentes incorreladas
\(\mathbf{C_{XX}} = \mathrm{\mathbf{diag}}[\sigma_{X_1}^2\ldots \sigma_{X_M}^2]\)
y:

\[
\sigma_{Y}^2 = k_1^2\sigma_{X_1}^2 + \ldots + k_M\sigma_{X_M}^2
\]

    En el caso que las componentes del vector \(\mathbf{X}\) sean
independientes, y que los coeficiones de la combinación lineal sean
todos la unidad, \(\mathbf{k}=(1 \ldots 1)^\mathrm{T}\), puede
demostrarse que la función de densidad de probabilidad de \(Y\) es la
\textbf{convolución} de las funciones de densidad de probabilidad (en el
caso discreto, se trataría de la \textbf{convolución discreta} de las
funciones de masa de probabilidad):

\[
f_Y(y) = f_{X_1}(y)* \ldots * f_{X_M}(y)
\]

En el caso de las variables aleatorias \(X_1 \ldots X_M\) sean
Gaussianas, su convolución resulta también Gaussiana, con media
\(\mu_Y = \mu_1 + \ldots + \mu_M\). Como la independencia implica
incorrelación, por la varianza es
\(\sigma_{Y}^2 = \sigma_{X_1}^2 + \ldots + \sigma_{X_M}^2\). Por tanto
\(f_Y(y) = N(\mu_Y, \sigma_Y^2)\).

El \textbf{Teorema del Límite Central} establece que, cuando \(M\) es
muy grande, la función de densidad de la superposición de variables
aleatorias independientes tiende a una Gaussiana, independientemente de
la caracterización de cada una de ellas. Este resultado justifica el
abundante uso de Gaussianas en la práctica, pues frecuentemente nos
encontraremos con superposiciones aleatorias de muchos elementos
independientes.

    La transformación lineal vista anteriormente puede extenderse como
sigue:

\[
\mathbf{Y} = \left( \begin{array}{c}{Y_1} \\ \vdots \\ {Y_N}\end{array}\right) = 
\left[ \begin{array}{ccc} {k_{1,1}} & \ldots & {k_{1,M}} \\ 
\vdots & \ddots & \vdots \\ 
{k_{N,1}} & \ldots & {k_{N,M}}\end{array}\right]
\left( \begin{array}{c}{X_1} \\ \vdots \\ {X_M}\end{array}\right) + \left( \begin{array}{c}{m_1} \\ \vdots \\ {m_N}\end{array}\right) = 
\mathbf{K}^\mathrm{T}\mathbf{X} + \mathbf{m}
\]

Por tanto:

\[
\boldsymbol{\mu}_{\mathbf{Y}} \equiv E(\mathbf{Y}) = 
\mathbf{K}^\mathrm{T}\boldsymbol{\mu}_{\mathbf{X}} + \mathbf{m}
\]

\[
\mathbf{C_{YY}} = \mathbf{K}^\mathrm{T}\mathbf{C_{XX}}\mathbf{K}
\]

Podemos también obtener la matriz de covarianza cruzada entre el
resultado y los operandos:

\[
\mathbf{C_{YX}} = \mathbf{K}^\mathrm{T}\mathbf{C_{XX}}
\]


    % Add a bibliography block to the postdoc
    
    
    
    \end{document}
