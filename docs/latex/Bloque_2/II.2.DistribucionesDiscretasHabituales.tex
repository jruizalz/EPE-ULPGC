
% Default to the notebook output style

    


% Inherit from the specified cell style.




    
\documentclass[11pt]{article}

    
    
    \usepackage[T1]{fontenc}
    % Nicer default font (+ math font) than Computer Modern for most use cases
    \usepackage{mathpazo}

    % Basic figure setup, for now with no caption control since it's done
    % automatically by Pandoc (which extracts ![](path) syntax from Markdown).
    \usepackage{graphicx}
    % We will generate all images so they have a width \maxwidth. This means
    % that they will get their normal width if they fit onto the page, but
    % are scaled down if they would overflow the margins.
    \makeatletter
    \def\maxwidth{\ifdim\Gin@nat@width>\linewidth\linewidth
    \else\Gin@nat@width\fi}
    \makeatother
    \let\Oldincludegraphics\includegraphics
    % Set max figure width to be 80% of text width, for now hardcoded.
    \renewcommand{\includegraphics}[1]{\Oldincludegraphics[width=.8\maxwidth]{#1}}
    % Ensure that by default, figures have no caption (until we provide a
    % proper Figure object with a Caption API and a way to capture that
    % in the conversion process - todo).
    \usepackage{caption}
    \DeclareCaptionLabelFormat{nolabel}{}
    \captionsetup{labelformat=nolabel}

    \usepackage{adjustbox} % Used to constrain images to a maximum size 
    \usepackage{xcolor} % Allow colors to be defined
    \usepackage{enumerate} % Needed for markdown enumerations to work
    \usepackage{geometry} % Used to adjust the document margins
    \usepackage{amsmath} % Equations
    \usepackage{amssymb} % Equations
    \usepackage{textcomp} % defines textquotesingle
    % Hack from http://tex.stackexchange.com/a/47451/13684:
    \AtBeginDocument{%
        \def\PYZsq{\textquotesingle}% Upright quotes in Pygmentized code
    }
    \usepackage{upquote} % Upright quotes for verbatim code
    \usepackage{eurosym} % defines \euro
    \usepackage[mathletters]{ucs} % Extended unicode (utf-8) support
    \usepackage[utf8x]{inputenc} % Allow utf-8 characters in the tex document
    \usepackage{fancyvrb} % verbatim replacement that allows latex
    \usepackage{grffile} % extends the file name processing of package graphics 
                         % to support a larger range 
    % The hyperref package gives us a pdf with properly built
    % internal navigation ('pdf bookmarks' for the table of contents,
    % internal cross-reference links, web links for URLs, etc.)
    \usepackage{hyperref}
    \usepackage{longtable} % longtable support required by pandoc >1.10
    \usepackage{booktabs}  % table support for pandoc > 1.12.2
    \usepackage[inline]{enumitem} % IRkernel/repr support (it uses the enumerate* environment)
    \usepackage[normalem]{ulem} % ulem is needed to support strikethroughs (\sout)
                                % normalem makes italics be italics, not underlines
  
    % AÑADIDO POR JUAN
    \usepackage{mathrsfs}   
  
    
    % Colors for the hyperref package
    \definecolor{urlcolor}{rgb}{0,.145,.698}
    \definecolor{linkcolor}{rgb}{.71,0.21,0.01}
    \definecolor{citecolor}{rgb}{.12,.54,.11}

    % ANSI colors
    \definecolor{ansi-black}{HTML}{3E424D}
    \definecolor{ansi-black-intense}{HTML}{282C36}
    \definecolor{ansi-red}{HTML}{E75C58}
    \definecolor{ansi-red-intense}{HTML}{B22B31}
    \definecolor{ansi-green}{HTML}{00A250}
    \definecolor{ansi-green-intense}{HTML}{007427}
    \definecolor{ansi-yellow}{HTML}{DDB62B}
    \definecolor{ansi-yellow-intense}{HTML}{B27D12}
    \definecolor{ansi-blue}{HTML}{208FFB}
    \definecolor{ansi-blue-intense}{HTML}{0065CA}
    \definecolor{ansi-magenta}{HTML}{D160C4}
    \definecolor{ansi-magenta-intense}{HTML}{A03196}
    \definecolor{ansi-cyan}{HTML}{60C6C8}
    \definecolor{ansi-cyan-intense}{HTML}{258F8F}
    \definecolor{ansi-white}{HTML}{C5C1B4}
    \definecolor{ansi-white-intense}{HTML}{A1A6B2}

    % commands and environments needed by pandoc snippets
    % extracted from the output of `pandoc -s`
    \providecommand{\tightlist}{%
      \setlength{\itemsep}{0pt}\setlength{\parskip}{0pt}}
    \DefineVerbatimEnvironment{Highlighting}{Verbatim}{commandchars=\\\{\}}
    % Add ',fontsize=\small' for more characters per line
    \newenvironment{Shaded}{}{}
    \newcommand{\KeywordTok}[1]{\textcolor[rgb]{0.00,0.44,0.13}{\textbf{{#1}}}}
    \newcommand{\DataTypeTok}[1]{\textcolor[rgb]{0.56,0.13,0.00}{{#1}}}
    \newcommand{\DecValTok}[1]{\textcolor[rgb]{0.25,0.63,0.44}{{#1}}}
    \newcommand{\BaseNTok}[1]{\textcolor[rgb]{0.25,0.63,0.44}{{#1}}}
    \newcommand{\FloatTok}[1]{\textcolor[rgb]{0.25,0.63,0.44}{{#1}}}
    \newcommand{\CharTok}[1]{\textcolor[rgb]{0.25,0.44,0.63}{{#1}}}
    \newcommand{\StringTok}[1]{\textcolor[rgb]{0.25,0.44,0.63}{{#1}}}
    \newcommand{\CommentTok}[1]{\textcolor[rgb]{0.38,0.63,0.69}{\textit{{#1}}}}
    \newcommand{\OtherTok}[1]{\textcolor[rgb]{0.00,0.44,0.13}{{#1}}}
    \newcommand{\AlertTok}[1]{\textcolor[rgb]{1.00,0.00,0.00}{\textbf{{#1}}}}
    \newcommand{\FunctionTok}[1]{\textcolor[rgb]{0.02,0.16,0.49}{{#1}}}
    \newcommand{\RegionMarkerTok}[1]{{#1}}
    \newcommand{\ErrorTok}[1]{\textcolor[rgb]{1.00,0.00,0.00}{\textbf{{#1}}}}
    \newcommand{\NormalTok}[1]{{#1}}
    
    % Additional commands for more recent versions of Pandoc
    \newcommand{\ConstantTok}[1]{\textcolor[rgb]{0.53,0.00,0.00}{{#1}}}
    \newcommand{\SpecialCharTok}[1]{\textcolor[rgb]{0.25,0.44,0.63}{{#1}}}
    \newcommand{\VerbatimStringTok}[1]{\textcolor[rgb]{0.25,0.44,0.63}{{#1}}}
    \newcommand{\SpecialStringTok}[1]{\textcolor[rgb]{0.73,0.40,0.53}{{#1}}}
    \newcommand{\ImportTok}[1]{{#1}}
    \newcommand{\DocumentationTok}[1]{\textcolor[rgb]{0.73,0.13,0.13}{\textit{{#1}}}}
    \newcommand{\AnnotationTok}[1]{\textcolor[rgb]{0.38,0.63,0.69}{\textbf{\textit{{#1}}}}}
    \newcommand{\CommentVarTok}[1]{\textcolor[rgb]{0.38,0.63,0.69}{\textbf{\textit{{#1}}}}}
    \newcommand{\VariableTok}[1]{\textcolor[rgb]{0.10,0.09,0.49}{{#1}}}
    \newcommand{\ControlFlowTok}[1]{\textcolor[rgb]{0.00,0.44,0.13}{\textbf{{#1}}}}
    \newcommand{\OperatorTok}[1]{\textcolor[rgb]{0.40,0.40,0.40}{{#1}}}
    \newcommand{\BuiltInTok}[1]{{#1}}
    \newcommand{\ExtensionTok}[1]{{#1}}
    \newcommand{\PreprocessorTok}[1]{\textcolor[rgb]{0.74,0.48,0.00}{{#1}}}
    \newcommand{\AttributeTok}[1]{\textcolor[rgb]{0.49,0.56,0.16}{{#1}}}
    \newcommand{\InformationTok}[1]{\textcolor[rgb]{0.38,0.63,0.69}{\textbf{\textit{{#1}}}}}
    \newcommand{\WarningTok}[1]{\textcolor[rgb]{0.38,0.63,0.69}{\textbf{\textit{{#1}}}}}
    
    
    % Define a nice break command that doesn't care if a line doesn't already
    % exist.
    \def\br{\hspace*{\fill} \\* }
    % Math Jax compatability definitions
    \def\gt{>}
    \def\lt{<}
    % Document parameters
    \title{II.2 Distribuciones Discretas Habituales}
    

    % Pygments definitions
    
\makeatletter
\def\PY@reset{\let\PY@it=\relax \let\PY@bf=\relax%
    \let\PY@ul=\relax \let\PY@tc=\relax%
    \let\PY@bc=\relax \let\PY@ff=\relax}
\def\PY@tok#1{\csname PY@tok@#1\endcsname}
\def\PY@toks#1+{\ifx\relax#1\empty\else%
    \PY@tok{#1}\expandafter\PY@toks\fi}
\def\PY@do#1{\PY@bc{\PY@tc{\PY@ul{%
    \PY@it{\PY@bf{\PY@ff{#1}}}}}}}
\def\PY#1#2{\PY@reset\PY@toks#1+\relax+\PY@do{#2}}

\expandafter\def\csname PY@tok@w\endcsname{\def\PY@tc##1{\textcolor[rgb]{0.73,0.73,0.73}{##1}}}
\expandafter\def\csname PY@tok@c\endcsname{\let\PY@it=\textit\def\PY@tc##1{\textcolor[rgb]{0.25,0.50,0.50}{##1}}}
\expandafter\def\csname PY@tok@cp\endcsname{\def\PY@tc##1{\textcolor[rgb]{0.74,0.48,0.00}{##1}}}
\expandafter\def\csname PY@tok@k\endcsname{\let\PY@bf=\textbf\def\PY@tc##1{\textcolor[rgb]{0.00,0.50,0.00}{##1}}}
\expandafter\def\csname PY@tok@kp\endcsname{\def\PY@tc##1{\textcolor[rgb]{0.00,0.50,0.00}{##1}}}
\expandafter\def\csname PY@tok@kt\endcsname{\def\PY@tc##1{\textcolor[rgb]{0.69,0.00,0.25}{##1}}}
\expandafter\def\csname PY@tok@o\endcsname{\def\PY@tc##1{\textcolor[rgb]{0.40,0.40,0.40}{##1}}}
\expandafter\def\csname PY@tok@ow\endcsname{\let\PY@bf=\textbf\def\PY@tc##1{\textcolor[rgb]{0.67,0.13,1.00}{##1}}}
\expandafter\def\csname PY@tok@nb\endcsname{\def\PY@tc##1{\textcolor[rgb]{0.00,0.50,0.00}{##1}}}
\expandafter\def\csname PY@tok@nf\endcsname{\def\PY@tc##1{\textcolor[rgb]{0.00,0.00,1.00}{##1}}}
\expandafter\def\csname PY@tok@nc\endcsname{\let\PY@bf=\textbf\def\PY@tc##1{\textcolor[rgb]{0.00,0.00,1.00}{##1}}}
\expandafter\def\csname PY@tok@nn\endcsname{\let\PY@bf=\textbf\def\PY@tc##1{\textcolor[rgb]{0.00,0.00,1.00}{##1}}}
\expandafter\def\csname PY@tok@ne\endcsname{\let\PY@bf=\textbf\def\PY@tc##1{\textcolor[rgb]{0.82,0.25,0.23}{##1}}}
\expandafter\def\csname PY@tok@nv\endcsname{\def\PY@tc##1{\textcolor[rgb]{0.10,0.09,0.49}{##1}}}
\expandafter\def\csname PY@tok@no\endcsname{\def\PY@tc##1{\textcolor[rgb]{0.53,0.00,0.00}{##1}}}
\expandafter\def\csname PY@tok@nl\endcsname{\def\PY@tc##1{\textcolor[rgb]{0.63,0.63,0.00}{##1}}}
\expandafter\def\csname PY@tok@ni\endcsname{\let\PY@bf=\textbf\def\PY@tc##1{\textcolor[rgb]{0.60,0.60,0.60}{##1}}}
\expandafter\def\csname PY@tok@na\endcsname{\def\PY@tc##1{\textcolor[rgb]{0.49,0.56,0.16}{##1}}}
\expandafter\def\csname PY@tok@nt\endcsname{\let\PY@bf=\textbf\def\PY@tc##1{\textcolor[rgb]{0.00,0.50,0.00}{##1}}}
\expandafter\def\csname PY@tok@nd\endcsname{\def\PY@tc##1{\textcolor[rgb]{0.67,0.13,1.00}{##1}}}
\expandafter\def\csname PY@tok@s\endcsname{\def\PY@tc##1{\textcolor[rgb]{0.73,0.13,0.13}{##1}}}
\expandafter\def\csname PY@tok@sd\endcsname{\let\PY@it=\textit\def\PY@tc##1{\textcolor[rgb]{0.73,0.13,0.13}{##1}}}
\expandafter\def\csname PY@tok@si\endcsname{\let\PY@bf=\textbf\def\PY@tc##1{\textcolor[rgb]{0.73,0.40,0.53}{##1}}}
\expandafter\def\csname PY@tok@se\endcsname{\let\PY@bf=\textbf\def\PY@tc##1{\textcolor[rgb]{0.73,0.40,0.13}{##1}}}
\expandafter\def\csname PY@tok@sr\endcsname{\def\PY@tc##1{\textcolor[rgb]{0.73,0.40,0.53}{##1}}}
\expandafter\def\csname PY@tok@ss\endcsname{\def\PY@tc##1{\textcolor[rgb]{0.10,0.09,0.49}{##1}}}
\expandafter\def\csname PY@tok@sx\endcsname{\def\PY@tc##1{\textcolor[rgb]{0.00,0.50,0.00}{##1}}}
\expandafter\def\csname PY@tok@m\endcsname{\def\PY@tc##1{\textcolor[rgb]{0.40,0.40,0.40}{##1}}}
\expandafter\def\csname PY@tok@gh\endcsname{\let\PY@bf=\textbf\def\PY@tc##1{\textcolor[rgb]{0.00,0.00,0.50}{##1}}}
\expandafter\def\csname PY@tok@gu\endcsname{\let\PY@bf=\textbf\def\PY@tc##1{\textcolor[rgb]{0.50,0.00,0.50}{##1}}}
\expandafter\def\csname PY@tok@gd\endcsname{\def\PY@tc##1{\textcolor[rgb]{0.63,0.00,0.00}{##1}}}
\expandafter\def\csname PY@tok@gi\endcsname{\def\PY@tc##1{\textcolor[rgb]{0.00,0.63,0.00}{##1}}}
\expandafter\def\csname PY@tok@gr\endcsname{\def\PY@tc##1{\textcolor[rgb]{1.00,0.00,0.00}{##1}}}
\expandafter\def\csname PY@tok@ge\endcsname{\let\PY@it=\textit}
\expandafter\def\csname PY@tok@gs\endcsname{\let\PY@bf=\textbf}
\expandafter\def\csname PY@tok@gp\endcsname{\let\PY@bf=\textbf\def\PY@tc##1{\textcolor[rgb]{0.00,0.00,0.50}{##1}}}
\expandafter\def\csname PY@tok@go\endcsname{\def\PY@tc##1{\textcolor[rgb]{0.53,0.53,0.53}{##1}}}
\expandafter\def\csname PY@tok@gt\endcsname{\def\PY@tc##1{\textcolor[rgb]{0.00,0.27,0.87}{##1}}}
\expandafter\def\csname PY@tok@err\endcsname{\def\PY@bc##1{\setlength{\fboxsep}{0pt}\fcolorbox[rgb]{1.00,0.00,0.00}{1,1,1}{\strut ##1}}}
\expandafter\def\csname PY@tok@kc\endcsname{\let\PY@bf=\textbf\def\PY@tc##1{\textcolor[rgb]{0.00,0.50,0.00}{##1}}}
\expandafter\def\csname PY@tok@kd\endcsname{\let\PY@bf=\textbf\def\PY@tc##1{\textcolor[rgb]{0.00,0.50,0.00}{##1}}}
\expandafter\def\csname PY@tok@kn\endcsname{\let\PY@bf=\textbf\def\PY@tc##1{\textcolor[rgb]{0.00,0.50,0.00}{##1}}}
\expandafter\def\csname PY@tok@kr\endcsname{\let\PY@bf=\textbf\def\PY@tc##1{\textcolor[rgb]{0.00,0.50,0.00}{##1}}}
\expandafter\def\csname PY@tok@bp\endcsname{\def\PY@tc##1{\textcolor[rgb]{0.00,0.50,0.00}{##1}}}
\expandafter\def\csname PY@tok@fm\endcsname{\def\PY@tc##1{\textcolor[rgb]{0.00,0.00,1.00}{##1}}}
\expandafter\def\csname PY@tok@vc\endcsname{\def\PY@tc##1{\textcolor[rgb]{0.10,0.09,0.49}{##1}}}
\expandafter\def\csname PY@tok@vg\endcsname{\def\PY@tc##1{\textcolor[rgb]{0.10,0.09,0.49}{##1}}}
\expandafter\def\csname PY@tok@vi\endcsname{\def\PY@tc##1{\textcolor[rgb]{0.10,0.09,0.49}{##1}}}
\expandafter\def\csname PY@tok@vm\endcsname{\def\PY@tc##1{\textcolor[rgb]{0.10,0.09,0.49}{##1}}}
\expandafter\def\csname PY@tok@sa\endcsname{\def\PY@tc##1{\textcolor[rgb]{0.73,0.13,0.13}{##1}}}
\expandafter\def\csname PY@tok@sb\endcsname{\def\PY@tc##1{\textcolor[rgb]{0.73,0.13,0.13}{##1}}}
\expandafter\def\csname PY@tok@sc\endcsname{\def\PY@tc##1{\textcolor[rgb]{0.73,0.13,0.13}{##1}}}
\expandafter\def\csname PY@tok@dl\endcsname{\def\PY@tc##1{\textcolor[rgb]{0.73,0.13,0.13}{##1}}}
\expandafter\def\csname PY@tok@s2\endcsname{\def\PY@tc##1{\textcolor[rgb]{0.73,0.13,0.13}{##1}}}
\expandafter\def\csname PY@tok@sh\endcsname{\def\PY@tc##1{\textcolor[rgb]{0.73,0.13,0.13}{##1}}}
\expandafter\def\csname PY@tok@s1\endcsname{\def\PY@tc##1{\textcolor[rgb]{0.73,0.13,0.13}{##1}}}
\expandafter\def\csname PY@tok@mb\endcsname{\def\PY@tc##1{\textcolor[rgb]{0.40,0.40,0.40}{##1}}}
\expandafter\def\csname PY@tok@mf\endcsname{\def\PY@tc##1{\textcolor[rgb]{0.40,0.40,0.40}{##1}}}
\expandafter\def\csname PY@tok@mh\endcsname{\def\PY@tc##1{\textcolor[rgb]{0.40,0.40,0.40}{##1}}}
\expandafter\def\csname PY@tok@mi\endcsname{\def\PY@tc##1{\textcolor[rgb]{0.40,0.40,0.40}{##1}}}
\expandafter\def\csname PY@tok@il\endcsname{\def\PY@tc##1{\textcolor[rgb]{0.40,0.40,0.40}{##1}}}
\expandafter\def\csname PY@tok@mo\endcsname{\def\PY@tc##1{\textcolor[rgb]{0.40,0.40,0.40}{##1}}}
\expandafter\def\csname PY@tok@ch\endcsname{\let\PY@it=\textit\def\PY@tc##1{\textcolor[rgb]{0.25,0.50,0.50}{##1}}}
\expandafter\def\csname PY@tok@cm\endcsname{\let\PY@it=\textit\def\PY@tc##1{\textcolor[rgb]{0.25,0.50,0.50}{##1}}}
\expandafter\def\csname PY@tok@cpf\endcsname{\let\PY@it=\textit\def\PY@tc##1{\textcolor[rgb]{0.25,0.50,0.50}{##1}}}
\expandafter\def\csname PY@tok@c1\endcsname{\let\PY@it=\textit\def\PY@tc##1{\textcolor[rgb]{0.25,0.50,0.50}{##1}}}
\expandafter\def\csname PY@tok@cs\endcsname{\let\PY@it=\textit\def\PY@tc##1{\textcolor[rgb]{0.25,0.50,0.50}{##1}}}

\def\PYZbs{\char`\\}
\def\PYZus{\char`\_}
\def\PYZob{\char`\{}
\def\PYZcb{\char`\}}
\def\PYZca{\char`\^}
\def\PYZam{\char`\&}
\def\PYZlt{\char`\<}
\def\PYZgt{\char`\>}
\def\PYZsh{\char`\#}
\def\PYZpc{\char`\%}
\def\PYZdl{\char`\$}
\def\PYZhy{\char`\-}
\def\PYZsq{\char`\'}
\def\PYZdq{\char`\"}
\def\PYZti{\char`\~}
% for compatibility with earlier versions
\def\PYZat{@}
\def\PYZlb{[}
\def\PYZrb{]}
\makeatother


    % Exact colors from NB
    \definecolor{incolor}{rgb}{0.0, 0.0, 0.5}
    \definecolor{outcolor}{rgb}{0.545, 0.0, 0.0}



    
    % Prevent overflowing lines due to hard-to-break entities
    \sloppy 
    % Setup hyperref package
    \hypersetup{
      breaklinks=true,  % so long urls are correctly broken across lines
      colorlinks=true,
      urlcolor=urlcolor,
      linkcolor=linkcolor,
      citecolor=citecolor,
      }
    % Slightly bigger margins than the latex defaults
    
    \geometry{verbose,tmargin=1in,bmargin=1in,lmargin=1in,rmargin=1in}
    
    

    \begin{document}
    
    
    \maketitle
    
    \hypertarget{ii.-2-distribuciones-de-masa-de-probabilidad-habituales}{%
\section*{Distribuciones de Masa de Probabilidad
Habituales}\label{ii.-2-distribuciones-de-masa-de-probabilidad-habituales}}

\hypertarget{la-distribuciuxf3n-de-bernoulli}{%
\subsection*{La Distribución de
Bernoulli}\label{la-distribuciuxf3n-de-bernoulli}}

Está asociada a la ocurrencia de un evento \(A\) con probabilidad
\(p=P(A)\), construyéndose la variable aleatoria \(X\) asignando el
valor \(1\) cuando \(A\) acontece y \(0\) cuando no lo hace. Modela la
probabilidad de acierto de un suceso. Por tanto, la función de masa de
probailidad es:

\[
\begin{array}{c|cc|c}
X & 0 & 1\\
\hline
p_X(x_i) & q=1-p & p=P(A) & \sum_i p_X(x_i) = 1
\end{array}
\]

Podemos dar una forma funcional, haciendo además explícita en la
definición la dependencia con el parámetro \(p\):

\[Bernoulli(p) \equiv p_X(x_i;p)=p^{x_i}(1-p)^{1-x_i} \qquad x_i \in \{0,1\}\]

    Podemos calcular fácilmente los principales estadísticos de la
distribución:

\begin{align*}
\eta_X = E(X) &= 0(1-p)+1p &= p\\
E(X^2) &= 0^2(1-p)+1^2p &= p\\
\sigma_X^2=Var(X) &= E(X^2)-\eta_X^2 = p-p^2 &= p(1-p)
\end{align*}

En el lanzamiento de una moneda, tal vez trucada, el suceso ``sacar
cara'' (igualmente sería ``cruz'') se modela con una distribución de
Bernoulli, siendo \(P(cara)=p\).

    \begin{Verbatim}[commandchars=\\\{\}]
Probabilidad de acertar: p =  0.70 
Probabilidad de fallar:  q = 1-p =  0.30

    \end{Verbatim}

    \begin{center}
    \adjustimage{max size={0.9\linewidth}{0.9\paperheight}}{II.2.DistribucionesDiscretasHabituales_files/II.2.DistribucionesDiscretasHabituales_3_1.png}
    \end{center}
    { \hspace*{\fill} \\}
    
    \begin{Verbatim}[commandchars=\\\{\}]
Bernoulli(0.7):
	 Media: 0.7
	 Varianza: 0.21000000000000002
	 Valor Cuadrático Medio: 0.7

Ejemplo de realización (100 repeticiones):
[0 1 1 0 0 1 0 1 1 1 1 0 1 0 1 1 0 1 0 1 1 0 1 0 1 1 1 1 0 0 0 0 0 1 1 0 0
 1 1 1 1 1 1 1 1 1 1 1 0 1 0 1 0 0 1 0 1 1 0 0 0 1 0 0 0 1 1 1 1 1 1 1 1 1
 1 0 1 1 0 1 1 1 1 1 1 1 0 1 1 0 1 1 1 1 0 1 0 1 1 0]

Estimaciones:
Probabilidades:  [0.35 0.65]
Media:  0.65 
Varianza:  0.2275

    \end{Verbatim}

    \hypertarget{la-distribuciuxf3n-binomial}{%
\subsection*{La Distribución
Binomial}\label{la-distribuciuxf3n-binomial}}

Consideremos la repetición de \(n\) experimentos de Bernoulli
independientes. La distribución binomial modela la probabilidad de que
el suceso \(A\) acontezca \(x_i\) veces en los \(n\) intentos:

\[B(n,p)\equiv p_X(x_i; n,p) = \binom{n}{x_i}p^{x_i}(1-p)^{n-x_i} \qquad n \in \mathbb{N} \quad x_i \in \{0,1 \ldots n\}\]

Considerando que una variable aleatoria binomial puede considerarse suma
de \(n\) variables de Bernoulli independientes, se demuestra (se verá
cuando se vean varias variables aleatorias conjuntas) que la media y la
varianza son:

\begin{align*}
\eta_X = E(X) &= np\\
\sigma_X^2=Var(X) &= np(1-p)
\end{align*}

    \begin{center}
    \adjustimage{max size={0.9\linewidth}{0.9\paperheight}}{II.2.DistribucionesDiscretasHabituales_files/II.2.DistribucionesDiscretasHabituales_6_0.png}
    \end{center}
    { \hspace*{\fill} \\}
    
    \begin{center}
    \adjustimage{max size={0.9\linewidth}{0.9\paperheight}}{II.2.DistribucionesDiscretasHabituales_files/II.2.DistribucionesDiscretasHabituales_7_0.png}
    \end{center}
    { \hspace*{\fill} \\}
    
    \begin{Verbatim}[commandchars=\\\{\}]
B( 5 , 0.5 ):
 	Media:  2.5 	Varianza:  1.25
B( 10 , 0.5 ):
 	Media:  5.0 	Varianza:  2.5
B( 20 , 0.5 ):
 	Media:  10.0 	Varianza:  5.0
B( 5 , 0.1 ):
 	Media:  0.5 	Varianza:  0.45
B( 10 , 0.05 ):
 	Media:  0.5 	Varianza:  0.475
B( 20 , 0.025 ):
 	Media:  0.5 	Varianza:  0.4875

    \end{Verbatim}

    \hypertarget{la-distribuciuxf3n-poisson}{%
\subsection*{La Distribución Poisson}\label{la-distribuciuxf3n-poisson}}

La función de masa de probabilidad de la distribución de Poisson es:

\[Pois(\lambda)\equiv p_X(x_i;\lambda) = e^{-\lambda}\frac{\lambda^k}{k!}\]

Esta distribución surge cuando se hacen crecer el número de repeticiones
de una distribución Binomial, de modo que la probabilidad de éxito se
hace muy pequeña pero el producto de tal probabilidad y el número de
repeticiones se mantiene constante:

\[n \rightarrow \infty \begin{cases}
    p \rightarrow 0\\
    np \rightarrow \lambda
  \end{cases} \implies B(n,p) \rightarrow Pois(\lambda)\]

El valor medio y la varianza resultan ser: \(\eta_X=\sigma_X^2=\lambda\)

    \begin{center}
    \adjustimage{max size={0.9\linewidth}{0.9\paperheight}}{II.2.DistribucionesDiscretasHabituales_files/II.2.DistribucionesDiscretasHabituales_10_0.png}
    \end{center}
    { \hspace*{\fill} \\}
    
    Los \textbf{procesos puntuales de Poisson} son importantes en la
práctica. Considérese un intervalo de longitud \(\Delta\) en el que se
resaltan puntos de forma totalmente aleatoria con una densidad o tasa de
llegada de \(r\). El promedio de puntos que surgirán en el mismo es
\(\lambda = r\Delta\). El intervalo podría ser también una porción del
plano o del espacio, con una interpretación semejante para los puntos.

\begin{itemize}
\tightlist
\item
  El número de puntos en dos subintervalos disjuntos es independiente,
  por la aleatoriedad en su surgimiento o llegada.
\item
  El número de puntos del intervalo tiene una distribución
  \(Pois(r\Delta)\)
\item
  Como se verá más adelante, si el intervalo es lineal, el periodo de
  espera hasta el siguiente punto sigue una distribución exponencial
\end{itemize}

    \hypertarget{distribuciones-multinoulli-y-uniforme}{%
\subsection*{Distribuciones Multinoulli y
Uniforme}\label{distribuciones-multinoulli-y-uniforme}}

Se trata de la distribución genérica de una variable aleatoria discreta
que puede tomar \(N\) valores posibles. Un ejemplo es el dado trucado
visto anteriormente, en el que \(N=6\) y cada cara tiene una
probabilidad \(p_X(x_i)\), con \(x_i \in \{1, \ldots 6\}\).

\begin{itemize}
\tightlist
\item
  Un caso particular es cuando \(N=2\), \(x_i \in \{0,1\}\), que
  corresponde a la distribución de Bernoulli.
\item
  La \textbf{distribución uniforme} es el caso particular en que la
  variable aleatoria toma valores entre \(a\) y \(b\), todos con igual
  probabilidad. Esto es:
\end{itemize}

\[U(a,b)\equiv p_X(x_i;a,b)=1/N \qquad a < b \in \mathbb{Z} \qquad N=b-a+1\]

\[\eta_X = \frac{a+b}{2}\]

\[\sigma_X^2 = \frac{(b-a+1)^2-1}{12}\]

    \begin{center}
    \adjustimage{max size={0.9\linewidth}{0.9\paperheight}}{II.2.DistribucionesDiscretasHabituales_files/II.2.DistribucionesDiscretasHabituales_13_0.png}
    \end{center}
    { \hspace*{\fill} \\}
    
    \hypertarget{verosimilitud-versus-bayes}{%
\subsection*{Verosimilitud versus
Bayes}\label{verosimilitud-versus-bayes}}

Hemos visto varias distribuciones que dependen de parámteros. Cada valor
de tales parámetros determina una distribución diferente. La variación
de tales distribuciones es continua con tales parámetros:

\begin{itemize}
\tightlist
\item
  Bernoulli(p)
\item
  Binom(n, p)
\item
  Poisson(\(\lambda\))
\item
  U(a,b)
\end{itemize}

Un parámetro \(k\) discreto también podría indexar distribuciones
totalmente diferentes. Por ejemplo, \(k \in \{1,2,3\}\) indicaría tres
distribuciones distintas.

    A la vista de observaciones provenientes de alguna distribución
parameterizada según se ha indicado, ¿podemos infererir cuál es la
distribución que las ha generado? En otras palabras, ¿cuáles son los
parámetros?

Consideremos que la distribución se parametriza como
\(p_X(x_i; \theta)\).

La \textbf{función de verosimilitud} corresponde a la función de masa de
probabilidad evaluada en la observación, dejando variar el parámetro que
consideramos \textbf{determinista y desconocido}:

\[L(\theta) = p_X(x_i; \theta)\]

\textbf{Alternativamente}, puede pensarse que \(\theta\) es una
\textbf{variable aleatoria}. En tal caso consideraremos las probabilidad
de masa condicionada de la observación por el parámetro:
\[p_X(x_i/\theta)\]

    Tenemos dos opciones para determinar el parámetro \(\theta\) y, con
ello, la distribución que ha generado las observaciones:

\begin{itemize}
\tightlist
\item
  \textbf{MLE}: Principio de \textbf{máxima verosimilitud}: buscamos el
  valor de \(\theta\) que maximiza la versomilitud \(L(\theta)\)
\item
  \textbf{MAP}: Principio de \textbf{máximo a posteriori}: \emph{si
  disponemos de la distribución a priori} del parámetro \(\theta\) (de
  momento sólo consideramos una variable aleaotoria discreta) podemos
  utilizar el \textbf{Teorema de Bayes} para obtener la probabilidad a
  posteriori del parámetro dadas las observaciones, que será la función
  a maximizar:
\end{itemize}

\[p_X(\theta/x_i) = \frac{p_X(x_i/\theta)p_\theta(\theta)}{\sum p_X(x_i/\theta_j)p_\theta(\theta_j)}\]

    \begin{center}
    \adjustimage{max size={0.9\linewidth}{0.9\paperheight}}{II.2.DistribucionesDiscretasHabituales_files/II.2.DistribucionesDiscretasHabituales_17_0.png}
    \end{center}
    { \hspace*{\fill} \\}
    
    \hypertarget{ejemplo}{%
\subsubsection*{Ejemplo}\label{ejemplo}}

Un sensor debe detectar dos tipos de radiaciones, ambas emitiendo
fotones conforme a una distribución de Poisson. La fuente 1 tiene
\(\lambda_1 = 5\) y la fuente 2 tiene \(\lambda_2 = 10\). Se miden 7
fotones.

\begin{itemize}
\tightlist
\item
  ¿Qué fuente debemos elegir?
\item
  ¿Y si sabemos que las fuentes tienen una probabilidad priori
  \(p_F(1)=1/3\) y \(p_F(2)=2/3\)?
\item
  ¿Cuáles deben ser las reglas de decisión MLE y MAP?
\item
  ¿Cuáles son las probabilidades de error y de acierto en cada caso?
\end{itemize}


    % Add a bibliography block to the postdoc
    
    
    
    \end{document}
