
% Default to the notebook output style

    


% Inherit from the specified cell style.




    
\documentclass[11pt]{article}

    
    
    \usepackage[T1]{fontenc}
    % Nicer default font (+ math font) than Computer Modern for most use cases
    \usepackage{mathpazo}

    % Basic figure setup, for now with no caption control since it's done
    % automatically by Pandoc (which extracts ![](path) syntax from Markdown).
    \usepackage{graphicx}
    % We will generate all images so they have a width \maxwidth. This means
    % that they will get their normal width if they fit onto the page, but
    % are scaled down if they would overflow the margins.
    \makeatletter
    \def\maxwidth{\ifdim\Gin@nat@width>\linewidth\linewidth
    \else\Gin@nat@width\fi}
    \makeatother
    \let\Oldincludegraphics\includegraphics
    % Set max figure width to be 80% of text width, for now hardcoded.
    \renewcommand{\includegraphics}[1]{\Oldincludegraphics[width=.8\maxwidth]{#1}}
    % Ensure that by default, figures have no caption (until we provide a
    % proper Figure object with a Caption API and a way to capture that
    % in the conversion process - todo).
    \usepackage{caption}
    \DeclareCaptionLabelFormat{nolabel}{}
    \captionsetup{labelformat=nolabel}

    \usepackage{adjustbox} % Used to constrain images to a maximum size 
    \usepackage{xcolor} % Allow colors to be defined
    \usepackage{enumerate} % Needed for markdown enumerations to work
    \usepackage{geometry} % Used to adjust the document margins
    \usepackage{amsmath} % Equations
    \usepackage{amssymb} % Equations
    \usepackage{textcomp} % defines textquotesingle
    % Hack from http://tex.stackexchange.com/a/47451/13684:
    \AtBeginDocument{%
        \def\PYZsq{\textquotesingle}% Upright quotes in Pygmentized code
    }
    \usepackage{upquote} % Upright quotes for verbatim code
    \usepackage{eurosym} % defines \euro
    \usepackage[mathletters]{ucs} % Extended unicode (utf-8) support
    \usepackage[utf8x]{inputenc} % Allow utf-8 characters in the tex document
    \usepackage{fancyvrb} % verbatim replacement that allows latex
    \usepackage{grffile} % extends the file name processing of package graphics 
                         % to support a larger range 
    % The hyperref package gives us a pdf with properly built
    % internal navigation ('pdf bookmarks' for the table of contents,
    % internal cross-reference links, web links for URLs, etc.)
    \usepackage{hyperref}
    \usepackage{longtable} % longtable support required by pandoc >1.10
    \usepackage{booktabs}  % table support for pandoc > 1.12.2
    \usepackage[inline]{enumitem} % IRkernel/repr support (it uses the enumerate* environment)
    \usepackage[normalem]{ulem} % ulem is needed to support strikethroughs (\sout)
                                % normalem makes italics be italics, not underlines
    
    % AÑADIDO POR JUAN
    \usepackage{mathrsfs}   
      
    
    % Colors for the hyperref package
    \definecolor{urlcolor}{rgb}{0,.145,.698}
    \definecolor{linkcolor}{rgb}{.71,0.21,0.01}
    \definecolor{citecolor}{rgb}{.12,.54,.11}

    % ANSI colors
    \definecolor{ansi-black}{HTML}{3E424D}
    \definecolor{ansi-black-intense}{HTML}{282C36}
    \definecolor{ansi-red}{HTML}{E75C58}
    \definecolor{ansi-red-intense}{HTML}{B22B31}
    \definecolor{ansi-green}{HTML}{00A250}
    \definecolor{ansi-green-intense}{HTML}{007427}
    \definecolor{ansi-yellow}{HTML}{DDB62B}
    \definecolor{ansi-yellow-intense}{HTML}{B27D12}
    \definecolor{ansi-blue}{HTML}{208FFB}
    \definecolor{ansi-blue-intense}{HTML}{0065CA}
    \definecolor{ansi-magenta}{HTML}{D160C4}
    \definecolor{ansi-magenta-intense}{HTML}{A03196}
    \definecolor{ansi-cyan}{HTML}{60C6C8}
    \definecolor{ansi-cyan-intense}{HTML}{258F8F}
    \definecolor{ansi-white}{HTML}{C5C1B4}
    \definecolor{ansi-white-intense}{HTML}{A1A6B2}

    % commands and environments needed by pandoc snippets
    % extracted from the output of `pandoc -s`
    \providecommand{\tightlist}{%
      \setlength{\itemsep}{0pt}\setlength{\parskip}{0pt}}
    \DefineVerbatimEnvironment{Highlighting}{Verbatim}{commandchars=\\\{\}}
    % Add ',fontsize=\small' for more characters per line
    \newenvironment{Shaded}{}{}
    \newcommand{\KeywordTok}[1]{\textcolor[rgb]{0.00,0.44,0.13}{\textbf{{#1}}}}
    \newcommand{\DataTypeTok}[1]{\textcolor[rgb]{0.56,0.13,0.00}{{#1}}}
    \newcommand{\DecValTok}[1]{\textcolor[rgb]{0.25,0.63,0.44}{{#1}}}
    \newcommand{\BaseNTok}[1]{\textcolor[rgb]{0.25,0.63,0.44}{{#1}}}
    \newcommand{\FloatTok}[1]{\textcolor[rgb]{0.25,0.63,0.44}{{#1}}}
    \newcommand{\CharTok}[1]{\textcolor[rgb]{0.25,0.44,0.63}{{#1}}}
    \newcommand{\StringTok}[1]{\textcolor[rgb]{0.25,0.44,0.63}{{#1}}}
    \newcommand{\CommentTok}[1]{\textcolor[rgb]{0.38,0.63,0.69}{\textit{{#1}}}}
    \newcommand{\OtherTok}[1]{\textcolor[rgb]{0.00,0.44,0.13}{{#1}}}
    \newcommand{\AlertTok}[1]{\textcolor[rgb]{1.00,0.00,0.00}{\textbf{{#1}}}}
    \newcommand{\FunctionTok}[1]{\textcolor[rgb]{0.02,0.16,0.49}{{#1}}}
    \newcommand{\RegionMarkerTok}[1]{{#1}}
    \newcommand{\ErrorTok}[1]{\textcolor[rgb]{1.00,0.00,0.00}{\textbf{{#1}}}}
    \newcommand{\NormalTok}[1]{{#1}}
    
    % Additional commands for more recent versions of Pandoc
    \newcommand{\ConstantTok}[1]{\textcolor[rgb]{0.53,0.00,0.00}{{#1}}}
    \newcommand{\SpecialCharTok}[1]{\textcolor[rgb]{0.25,0.44,0.63}{{#1}}}
    \newcommand{\VerbatimStringTok}[1]{\textcolor[rgb]{0.25,0.44,0.63}{{#1}}}
    \newcommand{\SpecialStringTok}[1]{\textcolor[rgb]{0.73,0.40,0.53}{{#1}}}
    \newcommand{\ImportTok}[1]{{#1}}
    \newcommand{\DocumentationTok}[1]{\textcolor[rgb]{0.73,0.13,0.13}{\textit{{#1}}}}
    \newcommand{\AnnotationTok}[1]{\textcolor[rgb]{0.38,0.63,0.69}{\textbf{\textit{{#1}}}}}
    \newcommand{\CommentVarTok}[1]{\textcolor[rgb]{0.38,0.63,0.69}{\textbf{\textit{{#1}}}}}
    \newcommand{\VariableTok}[1]{\textcolor[rgb]{0.10,0.09,0.49}{{#1}}}
    \newcommand{\ControlFlowTok}[1]{\textcolor[rgb]{0.00,0.44,0.13}{\textbf{{#1}}}}
    \newcommand{\OperatorTok}[1]{\textcolor[rgb]{0.40,0.40,0.40}{{#1}}}
    \newcommand{\BuiltInTok}[1]{{#1}}
    \newcommand{\ExtensionTok}[1]{{#1}}
    \newcommand{\PreprocessorTok}[1]{\textcolor[rgb]{0.74,0.48,0.00}{{#1}}}
    \newcommand{\AttributeTok}[1]{\textcolor[rgb]{0.49,0.56,0.16}{{#1}}}
    \newcommand{\InformationTok}[1]{\textcolor[rgb]{0.38,0.63,0.69}{\textbf{\textit{{#1}}}}}
    \newcommand{\WarningTok}[1]{\textcolor[rgb]{0.38,0.63,0.69}{\textbf{\textit{{#1}}}}}
    
    
    % Define a nice break command that doesn't care if a line doesn't already
    % exist.
    \def\br{\hspace*{\fill} \\* }
    % Math Jax compatability definitions
    \def\gt{>}
    \def\lt{<}
    % Document parameters
    \title{II.4 Distribuciones Continuas Habituales}
    
    
    

    % Pygments definitions
    
\makeatletter
\def\PY@reset{\let\PY@it=\relax \let\PY@bf=\relax%
    \let\PY@ul=\relax \let\PY@tc=\relax%
    \let\PY@bc=\relax \let\PY@ff=\relax}
\def\PY@tok#1{\csname PY@tok@#1\endcsname}
\def\PY@toks#1+{\ifx\relax#1\empty\else%
    \PY@tok{#1}\expandafter\PY@toks\fi}
\def\PY@do#1{\PY@bc{\PY@tc{\PY@ul{%
    \PY@it{\PY@bf{\PY@ff{#1}}}}}}}
\def\PY#1#2{\PY@reset\PY@toks#1+\relax+\PY@do{#2}}

\expandafter\def\csname PY@tok@w\endcsname{\def\PY@tc##1{\textcolor[rgb]{0.73,0.73,0.73}{##1}}}
\expandafter\def\csname PY@tok@c\endcsname{\let\PY@it=\textit\def\PY@tc##1{\textcolor[rgb]{0.25,0.50,0.50}{##1}}}
\expandafter\def\csname PY@tok@cp\endcsname{\def\PY@tc##1{\textcolor[rgb]{0.74,0.48,0.00}{##1}}}
\expandafter\def\csname PY@tok@k\endcsname{\let\PY@bf=\textbf\def\PY@tc##1{\textcolor[rgb]{0.00,0.50,0.00}{##1}}}
\expandafter\def\csname PY@tok@kp\endcsname{\def\PY@tc##1{\textcolor[rgb]{0.00,0.50,0.00}{##1}}}
\expandafter\def\csname PY@tok@kt\endcsname{\def\PY@tc##1{\textcolor[rgb]{0.69,0.00,0.25}{##1}}}
\expandafter\def\csname PY@tok@o\endcsname{\def\PY@tc##1{\textcolor[rgb]{0.40,0.40,0.40}{##1}}}
\expandafter\def\csname PY@tok@ow\endcsname{\let\PY@bf=\textbf\def\PY@tc##1{\textcolor[rgb]{0.67,0.13,1.00}{##1}}}
\expandafter\def\csname PY@tok@nb\endcsname{\def\PY@tc##1{\textcolor[rgb]{0.00,0.50,0.00}{##1}}}
\expandafter\def\csname PY@tok@nf\endcsname{\def\PY@tc##1{\textcolor[rgb]{0.00,0.00,1.00}{##1}}}
\expandafter\def\csname PY@tok@nc\endcsname{\let\PY@bf=\textbf\def\PY@tc##1{\textcolor[rgb]{0.00,0.00,1.00}{##1}}}
\expandafter\def\csname PY@tok@nn\endcsname{\let\PY@bf=\textbf\def\PY@tc##1{\textcolor[rgb]{0.00,0.00,1.00}{##1}}}
\expandafter\def\csname PY@tok@ne\endcsname{\let\PY@bf=\textbf\def\PY@tc##1{\textcolor[rgb]{0.82,0.25,0.23}{##1}}}
\expandafter\def\csname PY@tok@nv\endcsname{\def\PY@tc##1{\textcolor[rgb]{0.10,0.09,0.49}{##1}}}
\expandafter\def\csname PY@tok@no\endcsname{\def\PY@tc##1{\textcolor[rgb]{0.53,0.00,0.00}{##1}}}
\expandafter\def\csname PY@tok@nl\endcsname{\def\PY@tc##1{\textcolor[rgb]{0.63,0.63,0.00}{##1}}}
\expandafter\def\csname PY@tok@ni\endcsname{\let\PY@bf=\textbf\def\PY@tc##1{\textcolor[rgb]{0.60,0.60,0.60}{##1}}}
\expandafter\def\csname PY@tok@na\endcsname{\def\PY@tc##1{\textcolor[rgb]{0.49,0.56,0.16}{##1}}}
\expandafter\def\csname PY@tok@nt\endcsname{\let\PY@bf=\textbf\def\PY@tc##1{\textcolor[rgb]{0.00,0.50,0.00}{##1}}}
\expandafter\def\csname PY@tok@nd\endcsname{\def\PY@tc##1{\textcolor[rgb]{0.67,0.13,1.00}{##1}}}
\expandafter\def\csname PY@tok@s\endcsname{\def\PY@tc##1{\textcolor[rgb]{0.73,0.13,0.13}{##1}}}
\expandafter\def\csname PY@tok@sd\endcsname{\let\PY@it=\textit\def\PY@tc##1{\textcolor[rgb]{0.73,0.13,0.13}{##1}}}
\expandafter\def\csname PY@tok@si\endcsname{\let\PY@bf=\textbf\def\PY@tc##1{\textcolor[rgb]{0.73,0.40,0.53}{##1}}}
\expandafter\def\csname PY@tok@se\endcsname{\let\PY@bf=\textbf\def\PY@tc##1{\textcolor[rgb]{0.73,0.40,0.13}{##1}}}
\expandafter\def\csname PY@tok@sr\endcsname{\def\PY@tc##1{\textcolor[rgb]{0.73,0.40,0.53}{##1}}}
\expandafter\def\csname PY@tok@ss\endcsname{\def\PY@tc##1{\textcolor[rgb]{0.10,0.09,0.49}{##1}}}
\expandafter\def\csname PY@tok@sx\endcsname{\def\PY@tc##1{\textcolor[rgb]{0.00,0.50,0.00}{##1}}}
\expandafter\def\csname PY@tok@m\endcsname{\def\PY@tc##1{\textcolor[rgb]{0.40,0.40,0.40}{##1}}}
\expandafter\def\csname PY@tok@gh\endcsname{\let\PY@bf=\textbf\def\PY@tc##1{\textcolor[rgb]{0.00,0.00,0.50}{##1}}}
\expandafter\def\csname PY@tok@gu\endcsname{\let\PY@bf=\textbf\def\PY@tc##1{\textcolor[rgb]{0.50,0.00,0.50}{##1}}}
\expandafter\def\csname PY@tok@gd\endcsname{\def\PY@tc##1{\textcolor[rgb]{0.63,0.00,0.00}{##1}}}
\expandafter\def\csname PY@tok@gi\endcsname{\def\PY@tc##1{\textcolor[rgb]{0.00,0.63,0.00}{##1}}}
\expandafter\def\csname PY@tok@gr\endcsname{\def\PY@tc##1{\textcolor[rgb]{1.00,0.00,0.00}{##1}}}
\expandafter\def\csname PY@tok@ge\endcsname{\let\PY@it=\textit}
\expandafter\def\csname PY@tok@gs\endcsname{\let\PY@bf=\textbf}
\expandafter\def\csname PY@tok@gp\endcsname{\let\PY@bf=\textbf\def\PY@tc##1{\textcolor[rgb]{0.00,0.00,0.50}{##1}}}
\expandafter\def\csname PY@tok@go\endcsname{\def\PY@tc##1{\textcolor[rgb]{0.53,0.53,0.53}{##1}}}
\expandafter\def\csname PY@tok@gt\endcsname{\def\PY@tc##1{\textcolor[rgb]{0.00,0.27,0.87}{##1}}}
\expandafter\def\csname PY@tok@err\endcsname{\def\PY@bc##1{\setlength{\fboxsep}{0pt}\fcolorbox[rgb]{1.00,0.00,0.00}{1,1,1}{\strut ##1}}}
\expandafter\def\csname PY@tok@kc\endcsname{\let\PY@bf=\textbf\def\PY@tc##1{\textcolor[rgb]{0.00,0.50,0.00}{##1}}}
\expandafter\def\csname PY@tok@kd\endcsname{\let\PY@bf=\textbf\def\PY@tc##1{\textcolor[rgb]{0.00,0.50,0.00}{##1}}}
\expandafter\def\csname PY@tok@kn\endcsname{\let\PY@bf=\textbf\def\PY@tc##1{\textcolor[rgb]{0.00,0.50,0.00}{##1}}}
\expandafter\def\csname PY@tok@kr\endcsname{\let\PY@bf=\textbf\def\PY@tc##1{\textcolor[rgb]{0.00,0.50,0.00}{##1}}}
\expandafter\def\csname PY@tok@bp\endcsname{\def\PY@tc##1{\textcolor[rgb]{0.00,0.50,0.00}{##1}}}
\expandafter\def\csname PY@tok@fm\endcsname{\def\PY@tc##1{\textcolor[rgb]{0.00,0.00,1.00}{##1}}}
\expandafter\def\csname PY@tok@vc\endcsname{\def\PY@tc##1{\textcolor[rgb]{0.10,0.09,0.49}{##1}}}
\expandafter\def\csname PY@tok@vg\endcsname{\def\PY@tc##1{\textcolor[rgb]{0.10,0.09,0.49}{##1}}}
\expandafter\def\csname PY@tok@vi\endcsname{\def\PY@tc##1{\textcolor[rgb]{0.10,0.09,0.49}{##1}}}
\expandafter\def\csname PY@tok@vm\endcsname{\def\PY@tc##1{\textcolor[rgb]{0.10,0.09,0.49}{##1}}}
\expandafter\def\csname PY@tok@sa\endcsname{\def\PY@tc##1{\textcolor[rgb]{0.73,0.13,0.13}{##1}}}
\expandafter\def\csname PY@tok@sb\endcsname{\def\PY@tc##1{\textcolor[rgb]{0.73,0.13,0.13}{##1}}}
\expandafter\def\csname PY@tok@sc\endcsname{\def\PY@tc##1{\textcolor[rgb]{0.73,0.13,0.13}{##1}}}
\expandafter\def\csname PY@tok@dl\endcsname{\def\PY@tc##1{\textcolor[rgb]{0.73,0.13,0.13}{##1}}}
\expandafter\def\csname PY@tok@s2\endcsname{\def\PY@tc##1{\textcolor[rgb]{0.73,0.13,0.13}{##1}}}
\expandafter\def\csname PY@tok@sh\endcsname{\def\PY@tc##1{\textcolor[rgb]{0.73,0.13,0.13}{##1}}}
\expandafter\def\csname PY@tok@s1\endcsname{\def\PY@tc##1{\textcolor[rgb]{0.73,0.13,0.13}{##1}}}
\expandafter\def\csname PY@tok@mb\endcsname{\def\PY@tc##1{\textcolor[rgb]{0.40,0.40,0.40}{##1}}}
\expandafter\def\csname PY@tok@mf\endcsname{\def\PY@tc##1{\textcolor[rgb]{0.40,0.40,0.40}{##1}}}
\expandafter\def\csname PY@tok@mh\endcsname{\def\PY@tc##1{\textcolor[rgb]{0.40,0.40,0.40}{##1}}}
\expandafter\def\csname PY@tok@mi\endcsname{\def\PY@tc##1{\textcolor[rgb]{0.40,0.40,0.40}{##1}}}
\expandafter\def\csname PY@tok@il\endcsname{\def\PY@tc##1{\textcolor[rgb]{0.40,0.40,0.40}{##1}}}
\expandafter\def\csname PY@tok@mo\endcsname{\def\PY@tc##1{\textcolor[rgb]{0.40,0.40,0.40}{##1}}}
\expandafter\def\csname PY@tok@ch\endcsname{\let\PY@it=\textit\def\PY@tc##1{\textcolor[rgb]{0.25,0.50,0.50}{##1}}}
\expandafter\def\csname PY@tok@cm\endcsname{\let\PY@it=\textit\def\PY@tc##1{\textcolor[rgb]{0.25,0.50,0.50}{##1}}}
\expandafter\def\csname PY@tok@cpf\endcsname{\let\PY@it=\textit\def\PY@tc##1{\textcolor[rgb]{0.25,0.50,0.50}{##1}}}
\expandafter\def\csname PY@tok@c1\endcsname{\let\PY@it=\textit\def\PY@tc##1{\textcolor[rgb]{0.25,0.50,0.50}{##1}}}
\expandafter\def\csname PY@tok@cs\endcsname{\let\PY@it=\textit\def\PY@tc##1{\textcolor[rgb]{0.25,0.50,0.50}{##1}}}

\def\PYZbs{\char`\\}
\def\PYZus{\char`\_}
\def\PYZob{\char`\{}
\def\PYZcb{\char`\}}
\def\PYZca{\char`\^}
\def\PYZam{\char`\&}
\def\PYZlt{\char`\<}
\def\PYZgt{\char`\>}
\def\PYZsh{\char`\#}
\def\PYZpc{\char`\%}
\def\PYZdl{\char`\$}
\def\PYZhy{\char`\-}
\def\PYZsq{\char`\'}
\def\PYZdq{\char`\"}
\def\PYZti{\char`\~}
% for compatibility with earlier versions
\def\PYZat{@}
\def\PYZlb{[}
\def\PYZrb{]}
\makeatother


    % Exact colors from NB
    \definecolor{incolor}{rgb}{0.0, 0.0, 0.5}
    \definecolor{outcolor}{rgb}{0.545, 0.0, 0.0}



    
    % Prevent overflowing lines due to hard-to-break entities
    \sloppy 
    % Setup hyperref package
    \hypersetup{
      breaklinks=true,  % so long urls are correctly broken across lines
      colorlinks=true,
      urlcolor=urlcolor,
      linkcolor=linkcolor,
      citecolor=citecolor,
      }
    % Slightly bigger margins than the latex defaults
    
    \geometry{verbose,tmargin=1in,bmargin=1in,lmargin=1in,rmargin=1in}
    
    

    \begin{document}
    
    
    \maketitle
    
    

    
    \hypertarget{ii.4-variable-aleatoria-gaussiana-y-otras-habituales}{%
\section*{Variable Aleatoria Gaussiana y otras
habituales}\label{ii.4-variable-aleatoria-gaussiana-y-otras-habituales}}

\hypertarget{variable-aleatoria-uniforme}{%
\subsection*{Variable Aleatoria
Uniforme}\label{variable-aleatoria-uniforme}}

Denominamos \textbf{soporte de la variable aleatoria} a los valores de
la misma en los que la función de densidad de probabilidad toma valores
finitos distintos de cero. Considerando un intervalo de soporte
\(I = [a, b]\), se llama \textbf{variable aleatoria uniforme} a aquella
en que todos los intervalos de igual longitud y totalmente contenidos en
el intervalo de soporte sean \textbf{equiprobables}. La variable
aleatoria uniforme tiene una función de densidad de probabilidad:

\[
U(a,b) \equiv f_X(x;a,b) =
\begin{cases}
    \frac{1}{b-a} & \quad \text{si } a\leq x \leq b\\
    0             & \quad \text{resto }
\end{cases}
\]

Y función de distribución de probabilidad:

\[
F_X(x;a,b) =
\begin{cases}
    0                   & \quad \text{si } x < a \\
    \frac{1}{b-a} (x-a) & \quad \text{si } a\leq x \leq b\\
    1                   & \quad \text{si } x > b
\end{cases}
\]

    Calculando las integrales se obtienen fácilmente la media y la varianza:

\begin{itemize}
\tightlist
\item
  \(\eta_X = E(X) = \int_{-\infty}^{\infty} xf_X(x;a,b)= \frac{1}{b-a}\int_a^bxdx=\frac{a+b}{2}\)
\item
  \(E(X^2)= \int_{-\infty}^{\infty} x^2f_X(x;a,b)= \frac{1}{b-a}\int_a^bx^2dx=\frac{b^3-a^3}{3(b-a)}= \frac{a^2+ab+b^2}{3}\)
\item
  \(\sigma_X^2 = Var(X) = E((X-\eta_X)^2) = E(X^2)-\eta_X^2=\frac{(b-a)^2}{12}\)
\end{itemize}

    \begin{center}
    \adjustimage{max size={0.9\linewidth}{0.9\paperheight}}{II.4.DistribucionesContinuasHabituales_files/II.4.DistribucionesContinuasHabituales_3_0.png}
    \end{center}
    { \hspace*{\fill} \\}
    
    \hypertarget{funciones-condicionadas}{%
\subsubsection*{Funciones Condicionadas}\label{funciones-condicionadas}}

Consideremos el caso ya visto en que el suceso condicionante \(A\)
podemos definirlo con intervalos de la variable aleatoria. Suponiendo
que \(A=\{a \leq x_1\leq X \leq x_2 \leq b\}\). Adviértase que si
\(x_1 < a\) ó si \(x_2>b\), al extenderse el suceso condicionante fuera
del soporte de la distribución, \(A\) podría escribirse de forma
equivalente con \(x_1=a\) ó \(x_2=b\), respectivamente.

\begin{align*}
F_X(x\, /\, \{x_1\leq X \leq x_2\}) & = \frac{P(\{X \leq x \bigcap \{x_1\leq X \leq x_2\ )\} }{P(\{x_1\leq X \leq x_2\})}=\\
 & = \begin{cases}
    0       & \quad x < x_1\\
    \frac{F_X(x)-F_X(x_1^-)}{F_X(x_2)-F_X(x_1^-)}  & \quad x_1 \leq x \leq x_2\\
    1 &\quad x > x_2
  \end{cases}
\end{align*}

\begin{align*}
f_X(x\, /\, \{x_1\leq X \leq x_2\}) = \begin{cases}
    0       & \quad x < x_1\\
    \frac{f_X(x)}{F_X(x_2)-F_X(x_1^-)}  & \quad x_1 \leq x \leq x_2\\
    0 &\quad x > x_2
  \end{cases}
\end{align*}

    Sustituyendo para el caso concreto de la distribución uniforme, es fácil
obtener:

\[
f_X(x/A) =
\begin{cases}
    \frac{1}{x_2-x_1} & \quad \text{si } x_1\leq x \leq x_2\\
    0             & \quad \text{resto }
\end{cases}
\]

\[
F_X(x/A) =
\begin{cases}
    0                   & \quad \text{si } x < x_1 \\
    \frac{1}{x_2-x_1} (x-x_1) & \quad \text{si } x_1\leq x \leq x_2\\
    1                   & \quad \text{si } x > x_2
\end{cases}
\]

    \hypertarget{ejemplo}{%
\subsubsection*{Ejemplo}\label{ejemplo}}

Considérese que un suceso acontece con certeza en un intervalo, pero de
forma totalmente aleatoria dentro del mismo. Piénsese, por ejemplo, en
la llamada de una persona concreta que siempre se recibe diariamente
entre 22h y 23h.

Se pide:

\begin{enumerate}
\def\labelenumi{\arabic{enumi}.}
\tightlist
\item
  Modelar la VA \(X\) que representa el momento de recibir la llamada,
  obteniendo tanto la fdp como la FDP
\item
  Calcular la media y la varianza
\item
  Calcular la probabilidad de recibir la llamada entre las 22h y las
  22h:15m
\item
  Calcular la probabilidad de recibir la llamada justo a las 22h:15m
\item
  Si hasta las 22h:15m no se ha recibido la llamada, ¿cuál será la
  probabilidad de recibirla entre las 22hh:45m y las 23h?
\end{enumerate}

    \hypertarget{variable-aleatoria-exponencial}{%
\subsection*{Variable Aleatoria
Exponencial}\label{variable-aleatoria-exponencial}}

\textbf{La variable aleatoria exponencial modela el tiempo o espacio
entre el acontecimiento de dos sucesos sucesivos en un Proceso Puntual
de Poisson}. Tiene la siguiente función de densidad de probabilidad:

\[
exp(\lambda) \equiv f_X(x;\lambda) = \lambda e^{-\lambda x} u(x) =
\begin{cases}
    0             & \quad x<0\\
    \lambda e^{-\lambda x} & \quad x \geq 0
\end{cases} 
\]

Donde \(u(x)\) es la \textbf{función escalón}. La función de
distribución de probabilidad es:

\[
F_X(x;\lambda) = (1-e^{-\lambda x})u(x) = 
\begin{cases}
    0                   & \quad \text{si } x < 0 \\
    1-e^{-\lambda x} & \quad \text{si } x \geq 0
\end{cases}
\]

\(\lambda > 0\) se denomina \textbf{tasa}, si bien a veces se utiliza la
\textbf{escala} \(\beta = \lambda^{-1}\).

    Calculando las integrales por partes se obtienen la media y la varianza:

\begin{itemize}
\tightlist
\item
  \(\eta_X = E(X) = \int_{-\infty}^{\infty} xf_X(x;\lambda)= \lambda \int_0^\infty xe^{-\lambda u}du=\frac{1}{\lambda}=\beta\)
\item
  \(E(X^2)= \int_{-\infty}^{\infty} x^2f_X(x;\lambda)= \lambda \int_0^\infty x^2e^{-\lambda u}du=\frac{2}{\lambda^2}=2\beta^2\)
\item
  \(\sigma_X^2 = Var(X) = E((X-\eta_X)^2) = E(X^2)-\eta_X^2=\frac{1}{\lambda^2}=\beta^2\)
\end{itemize}

    \begin{center}
    \adjustimage{max size={0.9\linewidth}{0.9\paperheight}}{II.4.DistribucionesContinuasHabituales_files/II.4.DistribucionesContinuasHabituales_9_0.png}
    \end{center}
    { \hspace*{\fill} \\}
    
    \hypertarget{funciones-condicionadas-y-propiedad-sin-memoria}{%
\subsubsection*{Funciones condicionadas y propiedad sin
memoria}\label{funciones-condicionadas-y-propiedad-sin-memoria}}

Considérese el suceso condicionante \(A=\{X>x_1\}\). Las probabilidades
condicionadas se obtienen haciendo \(x_2=\infty\) en las ecuaciones ya
vistas:

\[
F_X(x/X>x_1) = \frac{F_X(x)-F_X(x_1)}{1-F_X(x_1)}u(x-x_1)
\]

\[
f_X(x/X>x_1) = \frac{f_X(x)}{1-F_X(x_1)}
\]

    Sustituyendo las expresiones correspondientes a la distribución
exponencial, resulta:

\[
F_X(x/X>x_1) = \left( 1-e^{-\lambda (x-x_1)}\right)u(x-x_1) = F_X(x-x_1)
\]

\[
f_X(x/X>x_1) = e^{-\lambda (x-x_1)}u(x-x_1) = f_X(x-x_1)
\]

Como puede verse, \textbf{la distribución exponencial no tiene memoria},
esto es, transcurrido un lapso \(x_1\) desde el último suceso la
distribución se comporta como si acabara de ocurrir, desplazándose hasta
\(x_1\).

    \hypertarget{ejemplo}{%
\subsubsection*{Ejemplo}\label{ejemplo}}

Considérese que un servidor informático recibe peticiones independientes
de clientes con una tasa de llegada \(\lambda = 2\) cada segundo. La
llegada de peticiones a lo largo del tiempo puede modelarse como un
Proceso Puntual de Poisson por ser las llegadas independientes.

\begin{itemize}
\tightlist
\item
  Si acaba de recibirse una petición, ¿cuál será la probabilidad de
  recibir otra en el próximo medio segundo?
\item
  ¿Y cuál será la probabilidad de no recibir otra hasta dentro de un
  segundo?
\item
  Tras la última llegada de una petición, extrañamente no ha entrado
  ninguna durante diez segundos. ¿Cuál será la probabilidad de que lo
  haga en el medio segundo siguiente?
\end{itemize}

    \hypertarget{la-variable-aleatoria-gaussiana}{%
\subsection*{La Variable Aleatoria
Gaussiana}\label{la-variable-aleatoria-gaussiana}}

\hypertarget{distribuciuxf3n-normal-estuxe1ndar}{%
\subsubsection*{Distribución Normal
Estándar}\label{distribuciuxf3n-normal-estuxe1ndar}}

Consideremos la función \(\varphi(x)\), definida como sigue:

\[\varphi_X(x) = \frac{1}{\sqrt{2\pi}}e^{-\frac{1}{2}x^2}\]

Puede advertirse que se trata de una función de densidad de
probabilidad, pues:

\[\varphi(x) >0 \quad \forall x \in \mathbb{R}\\
\int_{-\infty}^\infty \varphi(x)dx = 1\]

    Analicemos la función \(\varphi_X(x)\), que denominamos \textbf{Normal
Estándar}:

\begin{itemize}
\tightlist
\item
  Es una función par pues \(\varphi_X(x) = \varphi_X(-x)\)
\item
  \(\varphi'_X(x) = -x\varphi_X(x) \implies \varphi'_X(0) = 0\)
\item
  \(\varphi''_X(x) = (x^2-1)\varphi_X(x) \implies \varphi''_X(-1) = \varphi''_X(1) = 0\)
\item
  Máximo: \(\varphi_X(0) = \frac{1}{\sqrt{2\pi}}\approx 0.399\). Puntos
  de inflexión:
  \(\varphi_X(-1) = \varphi_X(1) =\frac{1}{\sqrt{2\pi e}}\approx 0.242\)
\end{itemize}

    \begin{center}
    \adjustimage{max size={0.9\linewidth}{0.9\paperheight}}{II.4.DistribucionesContinuasHabituales_files/II.4.DistribucionesContinuasHabituales_15_0.png}
    \end{center}
    { \hspace*{\fill} \\}
    
    La \textbf{función de distribución normal estándar} es:

\[\Phi(x) = \frac{1}{\sqrt{2\pi}} \int_{-\infty}^x  e^{-\frac{u^2}{2}} du = 
\frac{1}{2} + \frac{1}{\sqrt{2\pi}} \int_{0}^x  e^{-\frac{u^2}{2}} du
\]

En \emph{Teoría de la Comunicación} suele utilizarse la función
\(Q(x)\), que es el complemento a 1 de \(\Phi(x)\):

\[Q(x) = 1-\Phi(x) = \frac{1}{\sqrt{2\pi}} \int_x^{\infty}  e^{-\frac{u^2}{2}} du\]

    En ingeniería y ciencias es habitual utilizar la \textbf{función error},
\(erf(x)\):

\[erf(x) = \frac{2}{\sqrt{\pi}}\int_0^x e^{-u^2}du = \frac{1}{\sqrt{\pi}}\int_{-x}^x e^{-u^2}du= \frac{1}{\sqrt{2\pi}}\int_{-\sqrt{2}x}^{\sqrt{2}x} e^{-\frac{u^2}{2}}du\]

que satisface:

\begin{itemize}
\tightlist
\item
  Función impar: \(erf(x)=-erf(-x)\)
\item
  \(erf(\infty)=1\)
\item
  \(erf(1) \approx 0.8427\)
\end{itemize}

Su complemento a 1 es la \textbf{función error complementaria}:

\[erfc(x) = 1-erf(x) = \frac{2}{\sqrt{\pi}}\int_x^\infty e^{-u^2} du = \frac{2}{\sqrt{2\pi}}\int_{\sqrt{2}x}^\infty e^{-\frac{u^2}{2}}du\]

    Pueden encontrarse fácilmente relaciones entre las funciones anteriores:

\[\Phi(x) = \frac{1}{2}\left[1 + \frac{2}{\sqrt{2\pi}} \int_{0}^x  e^{-\frac{u^2}{2}}du \right] = \frac{1}{2}\left[1 + erf(\frac{x}{\sqrt{2}}) \right]\]

\[Q(x) = 1 - \Phi(x) = \frac{1}{2}\left[1 - erf(\frac{x}{\sqrt{2}}) \right] = \frac{1}{2}erfc(\frac{x}{\sqrt{2}})\]

Y puede calcularse la probabilidad de los intervalos:

\[P(x_1<X\leq x_2) = \Phi(x_2)-\Phi(x_1) = \frac{1}{2}erf(\frac{x_2}{\sqrt{2}}) - \frac{1}{2}erf(\frac{x_1}{\sqrt{2}})\]

En el caso que el intervalo esté centrado en el origen \(x_1 = -x_2\):

\[P(-x_2<X\leq x_2) = \Phi(x_2)-\Phi(x_1) = erf(\frac{x_2}{\sqrt{2}})\]

    \begin{center}
    \adjustimage{max size={0.9\linewidth}{0.9\paperheight}}{II.4.DistribucionesContinuasHabituales_files/II.4.DistribucionesContinuasHabituales_19_0.png}
    \end{center}
    { \hspace*{\fill} \\}
    
    Puede calcularse fácilmente la concentración de masa de probabilidad en
torno al origen:

\begin{itemize}
\tightlist
\item
  \(P(|X|\leq 1) = erf(\frac{1}{\sqrt2}) \approx 0.682689492137086\)
\item
  \(P(|X|\leq 2) = erf(\frac{2}{\sqrt2}) \approx 0.954499736103642\)
\item
  \(P(|X|\leq 3) = erf(\frac{3}{\sqrt2}) \approx 0.997300203936740\)
\end{itemize}

Adviértase, que al ser la fdp normal estándar una función par,
\(\varphi(x) = \varphi(-x)\), la esperanza matemática resulta nula. El
cálculo de la varianza requiere calcular la integral, y es igual a 1:

\begin{itemize}
\tightlist
\item
  \(\eta_X=E(X) = \int_\infty^\infty x\varphi(x) dx = 0\)
\item
  \(E(X^2) = \int_\infty^\infty x^2\varphi(x) dx = 1\)
\item
  \(\sigma_X^2=Var(X) = E(X^2) - E^2(X) = 1 \implies \sigma_X = 1\)
\end{itemize}

Por tanto, las concentraciones de masa de probabilidad anteriores están
calculadas a una, dos y tres \(\sigma\) de la media, respectivamente.

    \hypertarget{distribuciuxf3n-gaussiana-o-normal}{%
\subsubsection*{Distribución Gaussiana o
Normal}\label{distribuciuxf3n-gaussiana-o-normal}}

Si la variable aleatoria \(X\) tiene una distribución Normal Estándar
\(\varphi(x)\) (media nula y varianza unidad), podemos generar una nueva
variable aleatoria con media \(\eta_Y\) y varianza \(\sigma_Y^2\) sin
más que aplicar la transformación lineal:

\[Y = \sigma_Y X + \eta_Y\]

Puede demostrarse que la variable aleatoria \(Y\) tiene la siguiente
distribución:

\[F_Y(y)= \Phi(\frac{y-\eta_Y}{\sigma_Y})\]

\[f_Y(y) =\frac{dF_Y(y)}{dy} = \frac{1}{\sigma_Y}\varphi(\frac{y}{\sigma_Y})\]

    Intentemos ganar cierta intuición trabajando con estas transformaciones
de la función de densidad Normal Estándar. Adviértase que, en lo
sucesivo, tras los cambios de variable volvemos a llamar a la variable
\(X\). Aunque al principio puede resultar confuso, es habitual trabajar
así para tener las funciones de interés con \(x\) de variable
independiente.

La función \(\varphi(x)\) puede desplazarse en el eje real, y seguirá
siendo una función de densidad de probabilidad:

\[\varphi(x-\eta_X)=\frac{1}{\sqrt{2\pi}}e^{-\frac{1}{2}(x-\eta_X)^2}\]

Sin embargo, si le aplicamos un cambio de escala
\(\varphi(\frac{x}{\sigma_X})\) puede advertirse que el área bajo la
curva es \(\sigma_X\). Por ello, es necesario normalizar por el área
resultante tras el cambio de escala, para seguir teniendo una función de
densidad de probabilidad:
\[\frac{1}{\sigma_X}\varphi(\frac{x}{\sigma_X})\]

    \begin{center}
    \adjustimage{max size={0.9\linewidth}{0.9\paperheight}}{II.4.DistribucionesContinuasHabituales_files/II.4.DistribucionesContinuasHabituales_23_0.png}
    \end{center}
    { \hspace*{\fill} \\}
    
    \begin{center}
    \adjustimage{max size={0.9\linewidth}{0.9\paperheight}}{II.4.DistribucionesContinuasHabituales_files/II.4.DistribucionesContinuasHabituales_24_0.png}
    \end{center}
    { \hspace*{\fill} \\}
    
    Se define la distribución normal de parámetros \(\eta_X\) y \(\sigma_X\)
a la distribución de probabilidad con la siguiente función de densidad:

\[N(\eta_X, \sigma_X)\equiv f_X(x;\eta_X, \sigma_X) = \frac{1}{\sigma_X}\varphi(\frac{x-\eta_X}{\sigma_X})= \frac{1}{\sqrt{2\pi}\sigma_X}e^{-\frac{(x-\eta_X)^2}{2\sigma_X^2}}\]

Como se ha visto, \(E(X)=\eta_X\) y \(Var(X)=\sigma_X^2\).

La función de distribución de probabilidad es:

\begin{align*}
G(\eta_X, \sigma_X)&\equiv F_X(x; \eta_X, \sigma_X) = \Phi(\frac{x-\eta_X}{\sigma_X}) = \frac{1}{\sqrt{2\pi}\sigma_X}\int_{-\infty}^x  e^{-\frac{(u-\eta_X)^2}{2\sigma_X^2}} du=\\
&= \frac{1}{\sqrt{2\pi}\sigma_X}\int_{-\infty}^{\frac{x-\eta_X}{\sigma_X}}  e^{\frac{u^2}{2}} du= \frac{1}{2}\left[1 + erf(\frac{x-\eta_X}{\sigma_X\sqrt{2}}) \right]
\end{align*}

    \begin{center}
    \adjustimage{max size={0.9\linewidth}{0.9\paperheight}}{II.4.DistribucionesContinuasHabituales_files/II.4.DistribucionesContinuasHabituales_26_0.png}
    \end{center}
    { \hspace*{\fill} \\}
    
    \hypertarget{propiedades-asintuxf3ticas}{%
\subsubsection*{Propiedades
asintóticas}\label{propiedades-asintuxf3ticas}}

\hypertarget{aproximaciuxf3n-de-la-distribuciuxf3n-binomial}{%
\paragraph{Aproximación de la Distribución
Binomial}\label{aproximaciuxf3n-de-la-distribuciuxf3n-binomial}}

Conforme al \textbf{Teorema de Moivre-Laplace} la distribución Binomial
\(B(n,p)\) puede aproximarse por la distribución Normal
\(N(np, \sqrt(np(1-p)))\) cuando \(n\) es grande y \(p\) no está cerca
de \(0\) ni de \(1\). El Teorema prueba la convergencia en el límite. En
la práctica, se adopta la aproximación cuando \(n >20\) y
\(p \approx 0.5\). Cuanto mayor sea \(n\) más puede apartarse \(p\) del
valor \(0.5\), aunque nunca debe aproximarse a \(0\) ni \(1\).

\[B(n,p) \approx N(np, \sqrt(np(1-p)))\]

\hypertarget{aproximaciuxf3n-de-la-distribuciuxf3n-de-poisson}{%
\paragraph{Aproximación de la Distribución de
Poisson}\label{aproximaciuxf3n-de-la-distribuciuxf3n-de-poisson}}

La distribución de Poisson \(Pois(\lambda)\) puede aproximarse por la
distribución Normal \(N(\lambda, \lambda)\) cuando \(\lambda\) es grande

\[Pois(\lambda) \approx N(\lambda, \lambda)\]

    \hypertarget{funciuxf3n-de-densidad-condicionada}{%
\subsubsection*{Función de densidad
condicionada}\label{funciuxf3n-de-densidad-condicionada}}

Consideremos ahora el suceso condicionante \(A=\{x_1<X\leq x_2\}\).

\begin{align*}
F_X(x\, /\, \{x_1 < X \leq x_2\}) & = 
\begin{cases}
    0       & \quad x \leq x_1\\
    \frac{F_X(x)-F_X(x_1)}{F_X(x_2)-F_X(x_1)}  & \quad x_1 < x \leq x_2\\
    1 &\quad x > x_2
  \end{cases}
\end{align*}

\begin{align*}
f_X(x\, /\, \{x_1 < X \leq x_2\}) = \begin{cases}
    0       & \quad x \leq x_1\\
    \frac{f_X(x)}{F_X(x_2)-F_X(x_1)}  & \quad x_1 < x \leq x_2\\
    0 &\quad x > x_2
  \end{cases}
\end{align*}

    Sustituyendo para la variable aleatoria Gaussiana:

\[F_X(x_2)-F_X(x_1)=
\frac{1}{2}erf(\frac{x_2-\eta_X}{\sigma_X\sqrt{2}}) - \frac{1}{2}erf(\frac{x_1-\eta_X}{\sigma_X\sqrt{2}})
\]

Si \(x_1\) y \(x_2\) son simétricos respectos de \(\eta_X\):

\[F_X(x_2)-F_X(x_1)= erf(\frac{x_2-\eta_X}{\sigma_X\sqrt{2}})\]

Adviértase que, considerando este factor normalizante, podemos
reutilizar las expresiones de la distribución total (no condicionada)
para el cálculo de probabilidades en los subintervalos incluidos en el
suceso condicionante \(A\).

    \hypertarget{mezcla-de-gaussianas}{%
\subsubsection*{Mezcla de Gaussianas}\label{mezcla-de-gaussianas}}

Es habitual en la práctica confrontar distribuciones multimodales. Con
frecuencia se considera que estamos ante una superposición de
distribuciones Gaussianas, una por cada una de las modas existentes, con
sus respectivos valores medios y varianzas.

A veces conviene considerar que cada una de tales distribuciones
representa una subpoblación, y que cada observación de la variable
aleatoria proviene de una de ellas. Asociemos a cada subpoblación con
una hipótesis \(H_i\) de la que conocemos su probabilidad \emph{a
priori}, \(P(H_i)\). Consideremos que también conocemos las
verosimilitudes de las observaciones en relación a las subpoblaciones,
que podemos modelar mediante las probabilidades condicionadas:

\[f_{X_i}(x/H_i) \equiv N(\eta_{X_i}, \sigma_{X_i})\]

Podemos utilizar el Teorema de la Probabilidad Total para obtener la fdp
de la \textbf{mezcla de Gaussianas}:

\[f_X(x) = f_{X_1}(x/H_1)P(H_1)+ \ldots + f_{X_1}(x/H_1)P(H_1)\]

    \hypertarget{ejemplo}{%
\subsubsection*{Ejemplo}\label{ejemplo}}

En una población, el \(48\%\) de las personas son hombres y el resto
mujeres. Las estaturas de cada subpoblación se modela con una
distribución Gaussiana, de la siguiente manera. La estatura media de las
mujeres es de 165 cm y su desviación típica de 8 cm, mientras que la
estatura media de los hombres es de 175 cm y su desviación típica es de
10 cm.

\begin{itemize}
\tightlist
\item
  Calcule cuál es la probabilidad de que un hombre mida más de 2 m.
  Igualmente con una mujer
\item
  Considere ahora subpoblaciones de baloncestistas. En la femenina todas
  las estaturas son superiores a 170 cm, mientras que en las masculinas
  son todas superiores a 180 cm. ¿Cuál es la probabilidad de que una
  jugadora mida más de 2 m? ¿Y un jugador?
\item
  Suponga que tenemos una persona cuyo género desconocemos y que mide
  170 cm. ¿Qué es más verosímil, que sea un hombre o una mujer?
\item
  Obtenga utilizando el Teorema de la Probabilidad Total la mezcla de
  Gaussianas que modela la distribución total de hombres y mujeres
\item
  Utilice el Teorema de Bayes para ver qué es más probable, que la
  persona de 170 cm sea un hombre o una mujer.
\item
  ¿Cuál es la probabilidad de error si decide que es un hombre? ¿Y si
  decide que es una mujer?
\item
  Elabore una regla de decisión que permita seleccionar el género por
  estatura minimizando el error. ¿Cuál será la probabilidad de error de
  esta regla?
\end{itemize}

    % Add a bibliography block to the postdoc
    
    
    
    \end{document}
