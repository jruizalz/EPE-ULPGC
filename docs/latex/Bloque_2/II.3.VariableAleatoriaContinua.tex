
% Default to the notebook output style

    


% Inherit from the specified cell style.




    
\documentclass[11pt]{article}

    
    
    \usepackage[T1]{fontenc}
    % Nicer default font (+ math font) than Computer Modern for most use cases
    \usepackage{mathpazo}

    % Basic figure setup, for now with no caption control since it's done
    % automatically by Pandoc (which extracts ![](path) syntax from Markdown).
    \usepackage{graphicx}
    % We will generate all images so they have a width \maxwidth. This means
    % that they will get their normal width if they fit onto the page, but
    % are scaled down if they would overflow the margins.
    \makeatletter
    \def\maxwidth{\ifdim\Gin@nat@width>\linewidth\linewidth
    \else\Gin@nat@width\fi}
    \makeatother
    \let\Oldincludegraphics\includegraphics
    % Set max figure width to be 80% of text width, for now hardcoded.
    \renewcommand{\includegraphics}[1]{\Oldincludegraphics[width=.8\maxwidth]{#1}}
    % Ensure that by default, figures have no caption (until we provide a
    % proper Figure object with a Caption API and a way to capture that
    % in the conversion process - todo).
    \usepackage{caption}
    \DeclareCaptionLabelFormat{nolabel}{}
    \captionsetup{labelformat=nolabel}

    \usepackage{adjustbox} % Used to constrain images to a maximum size 
    \usepackage{xcolor} % Allow colors to be defined
    \usepackage{enumerate} % Needed for markdown enumerations to work
    \usepackage{geometry} % Used to adjust the document margins
    \usepackage{amsmath} % Equations
    \usepackage{amssymb} % Equations
    \usepackage{textcomp} % defines textquotesingle
    % Hack from http://tex.stackexchange.com/a/47451/13684:
    \AtBeginDocument{%
        \def\PYZsq{\textquotesingle}% Upright quotes in Pygmentized code
    }
    \usepackage{upquote} % Upright quotes for verbatim code
    \usepackage{eurosym} % defines \euro
    \usepackage[mathletters]{ucs} % Extended unicode (utf-8) support
    \usepackage[utf8x]{inputenc} % Allow utf-8 characters in the tex document
    \usepackage{fancyvrb} % verbatim replacement that allows latex
    \usepackage{grffile} % extends the file name processing of package graphics 
                         % to support a larger range 
    % The hyperref package gives us a pdf with properly built
    % internal navigation ('pdf bookmarks' for the table of contents,
    % internal cross-reference links, web links for URLs, etc.)
    \usepackage{hyperref}
    \usepackage{longtable} % longtable support required by pandoc >1.10
    \usepackage{booktabs}  % table support for pandoc > 1.12.2
    \usepackage[inline]{enumitem} % IRkernel/repr support (it uses the enumerate* environment)
    \usepackage[normalem]{ulem} % ulem is needed to support strikethroughs (\sout)
                                % normalem makes italics be italics, not underlines
 
    % AÑADIDO POR JUAN
    \usepackage{mathrsfs}   
        
    
    % Colors for the hyperref package
    \definecolor{urlcolor}{rgb}{0,.145,.698}
    \definecolor{linkcolor}{rgb}{.71,0.21,0.01}
    \definecolor{citecolor}{rgb}{.12,.54,.11}

    % ANSI colors
    \definecolor{ansi-black}{HTML}{3E424D}
    \definecolor{ansi-black-intense}{HTML}{282C36}
    \definecolor{ansi-red}{HTML}{E75C58}
    \definecolor{ansi-red-intense}{HTML}{B22B31}
    \definecolor{ansi-green}{HTML}{00A250}
    \definecolor{ansi-green-intense}{HTML}{007427}
    \definecolor{ansi-yellow}{HTML}{DDB62B}
    \definecolor{ansi-yellow-intense}{HTML}{B27D12}
    \definecolor{ansi-blue}{HTML}{208FFB}
    \definecolor{ansi-blue-intense}{HTML}{0065CA}
    \definecolor{ansi-magenta}{HTML}{D160C4}
    \definecolor{ansi-magenta-intense}{HTML}{A03196}
    \definecolor{ansi-cyan}{HTML}{60C6C8}
    \definecolor{ansi-cyan-intense}{HTML}{258F8F}
    \definecolor{ansi-white}{HTML}{C5C1B4}
    \definecolor{ansi-white-intense}{HTML}{A1A6B2}

    % commands and environments needed by pandoc snippets
    % extracted from the output of `pandoc -s`
    \providecommand{\tightlist}{%
      \setlength{\itemsep}{0pt}\setlength{\parskip}{0pt}}
    \DefineVerbatimEnvironment{Highlighting}{Verbatim}{commandchars=\\\{\}}
    % Add ',fontsize=\small' for more characters per line
    \newenvironment{Shaded}{}{}
    \newcommand{\KeywordTok}[1]{\textcolor[rgb]{0.00,0.44,0.13}{\textbf{{#1}}}}
    \newcommand{\DataTypeTok}[1]{\textcolor[rgb]{0.56,0.13,0.00}{{#1}}}
    \newcommand{\DecValTok}[1]{\textcolor[rgb]{0.25,0.63,0.44}{{#1}}}
    \newcommand{\BaseNTok}[1]{\textcolor[rgb]{0.25,0.63,0.44}{{#1}}}
    \newcommand{\FloatTok}[1]{\textcolor[rgb]{0.25,0.63,0.44}{{#1}}}
    \newcommand{\CharTok}[1]{\textcolor[rgb]{0.25,0.44,0.63}{{#1}}}
    \newcommand{\StringTok}[1]{\textcolor[rgb]{0.25,0.44,0.63}{{#1}}}
    \newcommand{\CommentTok}[1]{\textcolor[rgb]{0.38,0.63,0.69}{\textit{{#1}}}}
    \newcommand{\OtherTok}[1]{\textcolor[rgb]{0.00,0.44,0.13}{{#1}}}
    \newcommand{\AlertTok}[1]{\textcolor[rgb]{1.00,0.00,0.00}{\textbf{{#1}}}}
    \newcommand{\FunctionTok}[1]{\textcolor[rgb]{0.02,0.16,0.49}{{#1}}}
    \newcommand{\RegionMarkerTok}[1]{{#1}}
    \newcommand{\ErrorTok}[1]{\textcolor[rgb]{1.00,0.00,0.00}{\textbf{{#1}}}}
    \newcommand{\NormalTok}[1]{{#1}}
    
    % Additional commands for more recent versions of Pandoc
    \newcommand{\ConstantTok}[1]{\textcolor[rgb]{0.53,0.00,0.00}{{#1}}}
    \newcommand{\SpecialCharTok}[1]{\textcolor[rgb]{0.25,0.44,0.63}{{#1}}}
    \newcommand{\VerbatimStringTok}[1]{\textcolor[rgb]{0.25,0.44,0.63}{{#1}}}
    \newcommand{\SpecialStringTok}[1]{\textcolor[rgb]{0.73,0.40,0.53}{{#1}}}
    \newcommand{\ImportTok}[1]{{#1}}
    \newcommand{\DocumentationTok}[1]{\textcolor[rgb]{0.73,0.13,0.13}{\textit{{#1}}}}
    \newcommand{\AnnotationTok}[1]{\textcolor[rgb]{0.38,0.63,0.69}{\textbf{\textit{{#1}}}}}
    \newcommand{\CommentVarTok}[1]{\textcolor[rgb]{0.38,0.63,0.69}{\textbf{\textit{{#1}}}}}
    \newcommand{\VariableTok}[1]{\textcolor[rgb]{0.10,0.09,0.49}{{#1}}}
    \newcommand{\ControlFlowTok}[1]{\textcolor[rgb]{0.00,0.44,0.13}{\textbf{{#1}}}}
    \newcommand{\OperatorTok}[1]{\textcolor[rgb]{0.40,0.40,0.40}{{#1}}}
    \newcommand{\BuiltInTok}[1]{{#1}}
    \newcommand{\ExtensionTok}[1]{{#1}}
    \newcommand{\PreprocessorTok}[1]{\textcolor[rgb]{0.74,0.48,0.00}{{#1}}}
    \newcommand{\AttributeTok}[1]{\textcolor[rgb]{0.49,0.56,0.16}{{#1}}}
    \newcommand{\InformationTok}[1]{\textcolor[rgb]{0.38,0.63,0.69}{\textbf{\textit{{#1}}}}}
    \newcommand{\WarningTok}[1]{\textcolor[rgb]{0.38,0.63,0.69}{\textbf{\textit{{#1}}}}}
    
    
    % Define a nice break command that doesn't care if a line doesn't already
    % exist.
    \def\br{\hspace*{\fill} \\* }
    % Math Jax compatability definitions
    \def\gt{>}
    \def\lt{<}
    % Document parameters
    \title{II.3 Variable Aleatoria Continua}
    
    
  
    % Pygments definitions
    
\makeatletter
\def\PY@reset{\let\PY@it=\relax \let\PY@bf=\relax%
    \let\PY@ul=\relax \let\PY@tc=\relax%
    \let\PY@bc=\relax \let\PY@ff=\relax}
\def\PY@tok#1{\csname PY@tok@#1\endcsname}
\def\PY@toks#1+{\ifx\relax#1\empty\else%
    \PY@tok{#1}\expandafter\PY@toks\fi}
\def\PY@do#1{\PY@bc{\PY@tc{\PY@ul{%
    \PY@it{\PY@bf{\PY@ff{#1}}}}}}}
\def\PY#1#2{\PY@reset\PY@toks#1+\relax+\PY@do{#2}}

\expandafter\def\csname PY@tok@w\endcsname{\def\PY@tc##1{\textcolor[rgb]{0.73,0.73,0.73}{##1}}}
\expandafter\def\csname PY@tok@c\endcsname{\let\PY@it=\textit\def\PY@tc##1{\textcolor[rgb]{0.25,0.50,0.50}{##1}}}
\expandafter\def\csname PY@tok@cp\endcsname{\def\PY@tc##1{\textcolor[rgb]{0.74,0.48,0.00}{##1}}}
\expandafter\def\csname PY@tok@k\endcsname{\let\PY@bf=\textbf\def\PY@tc##1{\textcolor[rgb]{0.00,0.50,0.00}{##1}}}
\expandafter\def\csname PY@tok@kp\endcsname{\def\PY@tc##1{\textcolor[rgb]{0.00,0.50,0.00}{##1}}}
\expandafter\def\csname PY@tok@kt\endcsname{\def\PY@tc##1{\textcolor[rgb]{0.69,0.00,0.25}{##1}}}
\expandafter\def\csname PY@tok@o\endcsname{\def\PY@tc##1{\textcolor[rgb]{0.40,0.40,0.40}{##1}}}
\expandafter\def\csname PY@tok@ow\endcsname{\let\PY@bf=\textbf\def\PY@tc##1{\textcolor[rgb]{0.67,0.13,1.00}{##1}}}
\expandafter\def\csname PY@tok@nb\endcsname{\def\PY@tc##1{\textcolor[rgb]{0.00,0.50,0.00}{##1}}}
\expandafter\def\csname PY@tok@nf\endcsname{\def\PY@tc##1{\textcolor[rgb]{0.00,0.00,1.00}{##1}}}
\expandafter\def\csname PY@tok@nc\endcsname{\let\PY@bf=\textbf\def\PY@tc##1{\textcolor[rgb]{0.00,0.00,1.00}{##1}}}
\expandafter\def\csname PY@tok@nn\endcsname{\let\PY@bf=\textbf\def\PY@tc##1{\textcolor[rgb]{0.00,0.00,1.00}{##1}}}
\expandafter\def\csname PY@tok@ne\endcsname{\let\PY@bf=\textbf\def\PY@tc##1{\textcolor[rgb]{0.82,0.25,0.23}{##1}}}
\expandafter\def\csname PY@tok@nv\endcsname{\def\PY@tc##1{\textcolor[rgb]{0.10,0.09,0.49}{##1}}}
\expandafter\def\csname PY@tok@no\endcsname{\def\PY@tc##1{\textcolor[rgb]{0.53,0.00,0.00}{##1}}}
\expandafter\def\csname PY@tok@nl\endcsname{\def\PY@tc##1{\textcolor[rgb]{0.63,0.63,0.00}{##1}}}
\expandafter\def\csname PY@tok@ni\endcsname{\let\PY@bf=\textbf\def\PY@tc##1{\textcolor[rgb]{0.60,0.60,0.60}{##1}}}
\expandafter\def\csname PY@tok@na\endcsname{\def\PY@tc##1{\textcolor[rgb]{0.49,0.56,0.16}{##1}}}
\expandafter\def\csname PY@tok@nt\endcsname{\let\PY@bf=\textbf\def\PY@tc##1{\textcolor[rgb]{0.00,0.50,0.00}{##1}}}
\expandafter\def\csname PY@tok@nd\endcsname{\def\PY@tc##1{\textcolor[rgb]{0.67,0.13,1.00}{##1}}}
\expandafter\def\csname PY@tok@s\endcsname{\def\PY@tc##1{\textcolor[rgb]{0.73,0.13,0.13}{##1}}}
\expandafter\def\csname PY@tok@sd\endcsname{\let\PY@it=\textit\def\PY@tc##1{\textcolor[rgb]{0.73,0.13,0.13}{##1}}}
\expandafter\def\csname PY@tok@si\endcsname{\let\PY@bf=\textbf\def\PY@tc##1{\textcolor[rgb]{0.73,0.40,0.53}{##1}}}
\expandafter\def\csname PY@tok@se\endcsname{\let\PY@bf=\textbf\def\PY@tc##1{\textcolor[rgb]{0.73,0.40,0.13}{##1}}}
\expandafter\def\csname PY@tok@sr\endcsname{\def\PY@tc##1{\textcolor[rgb]{0.73,0.40,0.53}{##1}}}
\expandafter\def\csname PY@tok@ss\endcsname{\def\PY@tc##1{\textcolor[rgb]{0.10,0.09,0.49}{##1}}}
\expandafter\def\csname PY@tok@sx\endcsname{\def\PY@tc##1{\textcolor[rgb]{0.00,0.50,0.00}{##1}}}
\expandafter\def\csname PY@tok@m\endcsname{\def\PY@tc##1{\textcolor[rgb]{0.40,0.40,0.40}{##1}}}
\expandafter\def\csname PY@tok@gh\endcsname{\let\PY@bf=\textbf\def\PY@tc##1{\textcolor[rgb]{0.00,0.00,0.50}{##1}}}
\expandafter\def\csname PY@tok@gu\endcsname{\let\PY@bf=\textbf\def\PY@tc##1{\textcolor[rgb]{0.50,0.00,0.50}{##1}}}
\expandafter\def\csname PY@tok@gd\endcsname{\def\PY@tc##1{\textcolor[rgb]{0.63,0.00,0.00}{##1}}}
\expandafter\def\csname PY@tok@gi\endcsname{\def\PY@tc##1{\textcolor[rgb]{0.00,0.63,0.00}{##1}}}
\expandafter\def\csname PY@tok@gr\endcsname{\def\PY@tc##1{\textcolor[rgb]{1.00,0.00,0.00}{##1}}}
\expandafter\def\csname PY@tok@ge\endcsname{\let\PY@it=\textit}
\expandafter\def\csname PY@tok@gs\endcsname{\let\PY@bf=\textbf}
\expandafter\def\csname PY@tok@gp\endcsname{\let\PY@bf=\textbf\def\PY@tc##1{\textcolor[rgb]{0.00,0.00,0.50}{##1}}}
\expandafter\def\csname PY@tok@go\endcsname{\def\PY@tc##1{\textcolor[rgb]{0.53,0.53,0.53}{##1}}}
\expandafter\def\csname PY@tok@gt\endcsname{\def\PY@tc##1{\textcolor[rgb]{0.00,0.27,0.87}{##1}}}
\expandafter\def\csname PY@tok@err\endcsname{\def\PY@bc##1{\setlength{\fboxsep}{0pt}\fcolorbox[rgb]{1.00,0.00,0.00}{1,1,1}{\strut ##1}}}
\expandafter\def\csname PY@tok@kc\endcsname{\let\PY@bf=\textbf\def\PY@tc##1{\textcolor[rgb]{0.00,0.50,0.00}{##1}}}
\expandafter\def\csname PY@tok@kd\endcsname{\let\PY@bf=\textbf\def\PY@tc##1{\textcolor[rgb]{0.00,0.50,0.00}{##1}}}
\expandafter\def\csname PY@tok@kn\endcsname{\let\PY@bf=\textbf\def\PY@tc##1{\textcolor[rgb]{0.00,0.50,0.00}{##1}}}
\expandafter\def\csname PY@tok@kr\endcsname{\let\PY@bf=\textbf\def\PY@tc##1{\textcolor[rgb]{0.00,0.50,0.00}{##1}}}
\expandafter\def\csname PY@tok@bp\endcsname{\def\PY@tc##1{\textcolor[rgb]{0.00,0.50,0.00}{##1}}}
\expandafter\def\csname PY@tok@fm\endcsname{\def\PY@tc##1{\textcolor[rgb]{0.00,0.00,1.00}{##1}}}
\expandafter\def\csname PY@tok@vc\endcsname{\def\PY@tc##1{\textcolor[rgb]{0.10,0.09,0.49}{##1}}}
\expandafter\def\csname PY@tok@vg\endcsname{\def\PY@tc##1{\textcolor[rgb]{0.10,0.09,0.49}{##1}}}
\expandafter\def\csname PY@tok@vi\endcsname{\def\PY@tc##1{\textcolor[rgb]{0.10,0.09,0.49}{##1}}}
\expandafter\def\csname PY@tok@vm\endcsname{\def\PY@tc##1{\textcolor[rgb]{0.10,0.09,0.49}{##1}}}
\expandafter\def\csname PY@tok@sa\endcsname{\def\PY@tc##1{\textcolor[rgb]{0.73,0.13,0.13}{##1}}}
\expandafter\def\csname PY@tok@sb\endcsname{\def\PY@tc##1{\textcolor[rgb]{0.73,0.13,0.13}{##1}}}
\expandafter\def\csname PY@tok@sc\endcsname{\def\PY@tc##1{\textcolor[rgb]{0.73,0.13,0.13}{##1}}}
\expandafter\def\csname PY@tok@dl\endcsname{\def\PY@tc##1{\textcolor[rgb]{0.73,0.13,0.13}{##1}}}
\expandafter\def\csname PY@tok@s2\endcsname{\def\PY@tc##1{\textcolor[rgb]{0.73,0.13,0.13}{##1}}}
\expandafter\def\csname PY@tok@sh\endcsname{\def\PY@tc##1{\textcolor[rgb]{0.73,0.13,0.13}{##1}}}
\expandafter\def\csname PY@tok@s1\endcsname{\def\PY@tc##1{\textcolor[rgb]{0.73,0.13,0.13}{##1}}}
\expandafter\def\csname PY@tok@mb\endcsname{\def\PY@tc##1{\textcolor[rgb]{0.40,0.40,0.40}{##1}}}
\expandafter\def\csname PY@tok@mf\endcsname{\def\PY@tc##1{\textcolor[rgb]{0.40,0.40,0.40}{##1}}}
\expandafter\def\csname PY@tok@mh\endcsname{\def\PY@tc##1{\textcolor[rgb]{0.40,0.40,0.40}{##1}}}
\expandafter\def\csname PY@tok@mi\endcsname{\def\PY@tc##1{\textcolor[rgb]{0.40,0.40,0.40}{##1}}}
\expandafter\def\csname PY@tok@il\endcsname{\def\PY@tc##1{\textcolor[rgb]{0.40,0.40,0.40}{##1}}}
\expandafter\def\csname PY@tok@mo\endcsname{\def\PY@tc##1{\textcolor[rgb]{0.40,0.40,0.40}{##1}}}
\expandafter\def\csname PY@tok@ch\endcsname{\let\PY@it=\textit\def\PY@tc##1{\textcolor[rgb]{0.25,0.50,0.50}{##1}}}
\expandafter\def\csname PY@tok@cm\endcsname{\let\PY@it=\textit\def\PY@tc##1{\textcolor[rgb]{0.25,0.50,0.50}{##1}}}
\expandafter\def\csname PY@tok@cpf\endcsname{\let\PY@it=\textit\def\PY@tc##1{\textcolor[rgb]{0.25,0.50,0.50}{##1}}}
\expandafter\def\csname PY@tok@c1\endcsname{\let\PY@it=\textit\def\PY@tc##1{\textcolor[rgb]{0.25,0.50,0.50}{##1}}}
\expandafter\def\csname PY@tok@cs\endcsname{\let\PY@it=\textit\def\PY@tc##1{\textcolor[rgb]{0.25,0.50,0.50}{##1}}}

\def\PYZbs{\char`\\}
\def\PYZus{\char`\_}
\def\PYZob{\char`\{}
\def\PYZcb{\char`\}}
\def\PYZca{\char`\^}
\def\PYZam{\char`\&}
\def\PYZlt{\char`\<}
\def\PYZgt{\char`\>}
\def\PYZsh{\char`\#}
\def\PYZpc{\char`\%}
\def\PYZdl{\char`\$}
\def\PYZhy{\char`\-}
\def\PYZsq{\char`\'}
\def\PYZdq{\char`\"}
\def\PYZti{\char`\~}
% for compatibility with earlier versions
\def\PYZat{@}
\def\PYZlb{[}
\def\PYZrb{]}
\makeatother


    % Exact colors from NB
    \definecolor{incolor}{rgb}{0.0, 0.0, 0.5}
    \definecolor{outcolor}{rgb}{0.545, 0.0, 0.0}



    
    % Prevent overflowing lines due to hard-to-break entities
    \sloppy 
    % Setup hyperref package
    \hypersetup{
      breaklinks=true,  % so long urls are correctly broken across lines
      colorlinks=true,
      urlcolor=urlcolor,
      linkcolor=linkcolor,
      citecolor=citecolor,
      }
    % Slightly bigger margins than the latex defaults
    
    \geometry{verbose,tmargin=1in,bmargin=1in,lmargin=1in,rmargin=1in}
    
    

    \begin{document}
    
    
    \maketitle
    
    

    
    \hypertarget{ii.3-variable-aleatoria-continua}{%
\section*{Variable Aleatoria
Continua}\label{ii.3-variable-aleatoria-continua}}

Recordemos que, dado el \textbf{espacio de probabilidad}
\(\mathscr{E}(\Omega, \mathscr{F}, P)\), definimos la \textbf{variable
aleatoria} \(X\)

\begin{align*}
X:  \mathscr{\Omega} & \longmapsto \mathbb{R}\\
   \alpha & \longmapsto X(\alpha)
   \end{align*}

Consideremos ahora (a diferencia del caso discreto) que la variable
aleatoria \(X\) toma valores sobre el conjunto de los números reales
\(\mathbb{R}\), \emph{variando de forma continua la asignación de
probabilidades en intervalos de la recta real}. Esto hace que, en
general, las probabilidades asignadas a cada valor (pero no a cada
intervalo) de la variable aleatoria sean nulas:
\[P(X=x)=0 \quad \forall x \in \mathbb{R}\]

Es posible también que nos encontremos ante una \textbf{variable
aleatoria mixta}, llamada así por compartir características de las
discretas y de las continuas, en cuyo caso hay algunos
\(x_i \in \mathbb{R}\) con masas de probabilidad asignadas y
\(\sum_i P(x_i) < 1\) y el resto de la probabilidad se distribuye de
forma continua en intervalos.

    \hypertarget{funciuxf3n-de-distribuciuxf3n-de-probabilidad-fdp}{%
\subsection*{Función de distribución de probabilidad
(FDP)}\label{funciuxf3n-de-distribuciuxf3n-de-probabilidad-fdp}}

Puede demostrarse que cualquier suceso de \(\mathscr{F}\) puede
obtenerse mediante operaciones conjuntistas realizadas con sucesos
definidos sobre semi-intervalos de la forma \(\{X\leq x\}\), donde
\(x \in \mathbb{R}\) es un número real arbitrario. Si se tienen las
probabilidades de tales sucesos se conoce totalmente el espacio de
probabilidad de la variable aleatoria, lo que justifica la definición de
\textbf{función de distribución de probabilidad (FDP)} como:

\[F_X(x) = P(\{X\leq x\}) \quad -\infty \leq x \leq \infty\]

El \textbf{cuantil} \(x_q\) corresponde al valor que hace que el \(q\%\)
de la distribución esté por debajo. Por tanto,
\(q = F_X(x_q) \implies x_q = F_X^{-1}(q)\) y la \emph{función inversa}
\(F_X^{-1}\) se llama \textbf{función cuantil}. Por ejemplo,

\begin{itemize}
\tightlist
\item
  El primer cuartil es \(x_{0.25} = F_X^{-1}(0.25)\)
\item
  La mediana o segundo cuartil es \(x_{0.5} = F_X^{-1}(0.5)\)
\item
  El tercer cuartil es \(x_{0.75} = F_X^{-1}(0.75)\)
\item
  El octavo decil es \(x_{0.8} = F_X^{-1}(0.8)\)
\item
  El décimoquinto percentil es \(x_{0.15} = F_X^{-1}(0.15)\)
\end{itemize}

    \hypertarget{propiedades-de-la-funciuxf3n-de-distribuciuxf3n-de-probabilidad}{%
\subsubsection*{Propiedades de la función de distribución de
probabilidad}\label{propiedades-de-la-funciuxf3n-de-distribuciuxf3n-de-probabilidad}}

Sean \(F_X(x^-)=\lim \limits_{\epsilon \to 0} F_X(x-\epsilon)\) y
\(F_X(x^+)=\lim \limits_{\epsilon \to 0} F_X(x+\epsilon)\) los valores
de la función de distribución en \(x\) al acercarse respectivamente por
la izquierda y por la derecha. Se cumplen las siguientes propiedades:

\begin{enumerate}
\def\labelenumi{\arabic{enumi}.}
\tightlist
\item
  \(F_X(-\infty) = 0 \quad F_X(+\infty) = 1\)
\item
  \(P(\{X>x\})=1-F_X(x)\)
\item
  \(F_X\) es continua por la derecha: \(F_X(x^+)=F_X(x)\)
\item
  Si la variable aleatoria es continua, \(F_X\) también lo es y
  \(F_X(x^-)=F_X(x)\)
\item
  \(F_X\) es no decreciente:
  \(x_1 < x_2 \implies F_X(x_1)\leq F_X(x_2)\)
\item
  \(F_X(x_0)=0 \implies F_X(x) = 0 \quad\forall x \leq x_0\)
\item
  En los puntos de discontinuidad, \(P(\{X = x\})=F_X(x)-F_X(x^-)\)
\item
  \(P(\{x_1<X\leq x_2\})=F_X(x_2)-F_X(x_1)\)
\item
  \(P(\{x_1 \leq X\leq x_2\})=F_X(x_2)-F_X(x_1^-)\)
\item
  Si la \textbf{variable aleatoria es discreta}, la función de
  distribución es constante salvo en los puntos de discontinuidad en los
  que se sitúan los sucesos muestrales elementales.
\item
  Si la \textbf{variable aleatoria es mixta}, la función de distribución
  es continua salvo en los puntos en los que están localizadas las masas
  de probabilidad discreta, en los que la función presenta
  discontinuidades de salto igual a tales probabilidades.
\end{enumerate}

    \hypertarget{funciuxf3n-de-densidad-de-probabilidad-fdp}{%
\subsection*{Función de densidad de probabilidad
(fdp)}\label{funciuxf3n-de-densidad-de-probabilidad-fdp}}

Suele ser conveniente hacer un símil entre probabilidad y masa, de modo
que nuestra distribución de probabilidad podemos interpretarla como una
masa igualmente distribuida a lo largo del eje real.

Si la \textbf{variable es discreta}, tendremos \textbf{masas puntuales}
ideales en puntos del eje real, mientras que si la \textbf{variable es
continua}, la masa se distribuirá concentrándose más o menos a lo largo
del eje, pero sin llegar a formar masas puntuales. La \textbf{variable
mixta} combina ambos tipos de masa a lo largo del eje.

Como ya se ha visto, \textbf{la función de distribución describe la masa
acumulada hasta una posición} y, por diferencias, puede calcularse la
masa de cualquier intervalo.

La \textbf{función de densidad}, como su nombre indica, describe la
densidad de masa (masa por unidad de longitud) a lo largo del eje.

    La \textbf{función de densidad de probabilidad} (fdp) se define:

\[f_X(x) =\frac{dF_X(x)}{dx}=\lim \limits_{\Delta x \to 0} \frac{F_X(x+\Delta x/2)-F_X(x-\Delta x/2)}{\Delta x}= 
\lim \limits_{\Delta x \to 0} \frac{P\left(\{(x-\Delta x/2) < x \leq (x+\Delta x/2)\}\right)}{\Delta x}
\]

Si \textbf{la variable aleatoria es continua},
\[P\left(\{(x-\Delta x/2) < x \leq (x+\Delta x/2)\}\right) \approx f_X(x)\Delta x\]

Si \textbf{la variable aleatoria es discreta}, \(F_X(x)\) resulta
discontinua por la izquierda en los puntos donde se localizan masas de
probabilidad y no es derivable en los mismos. En tales puntos \({x_i}\)
puede definirse la \textbf{función de masa de probabilidad}
\(p_X(x_i)=P(\{X=x_i\})\), ya vista anteriormente.

Matemáticamente, puede usuarse la función \(\delta(x)\), que tiene
\textbf{área uno} localizada en \(x=0\), anchura nula y altura infinita
para relacionar las funciones de masa y de densidad de probabilidad. Si
la variable aleatoria es discreta:
\[f_X(x)=\sum_{i}p_X(x_i)\delta(x-x_i)\]

    \hypertarget{propiedades-de-la-funciuxf3n-de-densidad-de-probabilidad-fdp}{%
\subsubsection*{Propiedades de la función de densidad de probabilidad
(fdp)}\label{propiedades-de-la-funciuxf3n-de-densidad-de-probabilidad-fdp}}

La función de distribución de probabilidad (FDP) se obtiene integrando:

\[f_X(x) = \frac{dF_X(x)}{dx} \iff F_X(x)=\int_{-\infty}^x f_X(u)du\]

Dado que \(F_X(\infty)=1\) se concluye que
\(\int_{-\infty}^{\infty} f_X(x)dx = 1\).

Por tanto, el área debajo de la función de densidad de probabilidad
\textbf{siempre} es uno. Esto es lógico, pues el eje real corresponde al
espacio muestral o suceso seguro.

La probabilidad de un intervalo es
\[P(x_1< X \leq x_2) = F_X(x_2)-F_X(x_1)=\int_{x_1}^{x_2}f_X(x)dx\]

y corresponde al área debajo de la curva de la fdp en dicho intervalo.
Adviértase que \(\int_{0^-}^{0^+}\delta(x)dx = 1\), lo que nos permite
extender el formalismo a variables discretas y mixtas.

    \hypertarget{propiedades-de-la-funciuxf3n-de-masa-de-probabilidad-fmp}{%
\subsubsection*{Propiedades de la función de masa de probabilidad
(fmp)}\label{propiedades-de-la-funciuxf3n-de-masa-de-probabilidad-fmp}}

Como ya se ha visto, cuando la variable aleatoria es discreta, se define
la \textbf{función de masa de probabilidad}, que se relaciona como sigue
con la fdp:

\[p_X(x_i)=P(\{X=x_i\})\iff f_X(x)=\sum_{i}p_X(x_i)\delta(x-x_i)\]

Podemos relacionar la función de masa de probabilidad (fmp) con la
función de distribución de probabilidad (FDP), que corresponde a la
función acumulada de probabilidad vissta anteriormente:

\[p_X(x_n) = F_X(x_n)-F_X(x_{n-1}) \iff F_x(x_n)=\sum_{i=-\infty}^n p_X(x_i)\]

Siguen fácilmente las siguientes propiedades:

\[F_x(\infty)=1 \implies \sum_{i=-\infty}^{+\infty} p_X(x_i)=1\]

\[P(x_{n_1} \leq X \leq x_{n_2}) = \sum_{i=n_1}^{n_2} p_X(x_i)\]

    \hypertarget{relaciuxf3n-entre-el-histograma-y-la-fdp}{%
\paragraph{Relación entre el histograma y la
fdp}\label{relaciuxf3n-entre-el-histograma-y-la-fdp}}

Supongamos que tenemos una muestra estadística con \(N\) observaciones
de una variable estadística. Podemos considerar que tales observaciones
corresponden a realizaciones de un experimento modelado por un espacio
de probabilidad sobre el que hemos definido una variable aleatoria
correspondiente con la variable estadística.

El espacio de probabilidad propociona un modelo teórico de los datos,
mientras que las muestra estadísitca resulta de la experiencia empírica.
¿Podemos caracterizar probabilísticamente la población a partir de la
muetsra estadística?

Como sabemos podemos construir un histograma de frecuencias relativas,
sin más que agrupar los valores de la variable estadística por
intervalos. Considerando que todos los intervalos tienen igual anchura
\(\Delta x\), podemos aproximar la fdp en cada \(x_i\) (puntos medios
del intervalo, como sigue, siendo \(F_i\) la frecuencia absoluta de
observaciones que caen dentro del intervalo:

\[f_X(x_i) \approx \frac{F_i}{N\Delta x}\]

    \hypertarget{esperanza-matemuxe1tica-y-momentos}{%
\subsection*{Esperanza matemática y
momentos}\label{esperanza-matemuxe1tica-y-momentos}}

La esperanza matemática o media de una variable aleatoria \(X\) es:

\[\eta_X = E\{X\}=\int_{-\infty}^{\infty}xf_X(x)dx\]

Si consideramos la aproximación anterior del histograma a la fdp,
podemos ver que la media muestral aproxima la esperanza matemática:
\(\hat x = \sum_{i=1}^Nx_i\frac{F_i}{\Delta x} \approx \eta_X\).

La esperanza matemática puede definirse sobre cualquier función \(g(X)\)
de la variable aleatoria:

\[\eta_{g(X)} = E\{g(X)\}=\int_{-\infty}^{\infty}g(x)f_X(x)dx\]

La \textbf{esperanza matemática es un operador lineal}, esto es, sean
\(g_1(X)\) y \(g_2(X)\):

\[E\{a_1g_1(X)+a_2g_2(X)\}=a_1E\{g_1(X)\}+a_2E\{g_2(X)\}\]

    Podemos así definir los \textbf{momentos}:

\[m_n=E\{X^n\}=\int_{-\infty}^{\infty}x^nf_X(x)dx\]

En particular, la \textbf{media} es \(m_1\equiv\eta_X=E\{X^1\}\) y el
\textbf{valor cuadrático medio} es \(m_2=E\{X^2\}\)

Y los \textbf{momentos centrales}:

\[\mu_n=E\{(X-\eta_X)^n\}=\int_{-\infty}^{\infty}(x-\eta_X)^nf_X(x)dx\]

En particular, la \textbf{varianza} es
\(m_2\equiv \sigma_X^2 =E\{(X-\eta_X)^2\}\). Aplicando la linealidad:

\[\sigma_X^2=E\{(X-\eta_X)^2\}=E\{X^2\}-\eta_X^2\]

    \hypertarget{esperanza-y-momentos-en-el-caso-discreto}{%
\subsubsection*{Esperanza y momentos en el caso
discreto}\label{esperanza-y-momentos-en-el-caso-discreto}}

Como ya sabemos, si la \textbf{variable aleatoria es discreta}, se
utilizan las funciones de masa de probabilidad en vez de las fdp, y
sumas en vez de integrales, repitiéndose aquí las esperanzas y momentos
para el caso discreto vistos anteriormente:

\[\eta_{g(X)} = E\{g(X)\}=\sum_{i=-\infty}^{\infty}g(x_i)p_X(x_i)\]

\[m_n=E\{X^n\}=\sum_{i=-\infty}^{\infty}x^nf_X(x)dx\]

\[\mu_n=E\{(X-\eta_X)^n\}=\sum_{i=-\infty}^{\infty}(x-\eta_X)^nf_X(x)dx\]

    Todos estos momentos tienen sus equivalentes muestrales, ya vistos en la
parte de introducción a la estadística.

Por otra parte, es muy habitual \textbf{normalizar} una variable
aleatoria \(X\), construyendo otra \(Y=\frac{X-\eta_X}{\sigma_X}\) que
tenga media nula y varianza unidad, tal y como ya se ha visto para el
caso discreto.

Además, podemos definir:

\begin{itemize}
\tightlist
\item
  Coeficiente de asimetría:
  \(\gamma_1=E\{\frac{(X-\eta_X)^3}{\sigma^3}\}=\frac{\mu_3}{\sigma_X^3}\)
\item
  Coeficiente de exceso de curtosis:
  \(\gamma_2=E\{\frac{(X-\eta_X)^4}{\sigma^4}\}-3=\frac{\mu_4}{\sigma_X^4}-3\).
  La distribución normal tiene un exceso de curtosis de 0, razón por la
  que se resta 3.
\end{itemize}

    \hypertarget{funciones-condicionadas}{%
\subsection*{Funciones condicionadas}\label{funciones-condicionadas}}

Consideremos un espacio de probabilidad \(\mathscr{E}\) en el que se ha
definido una variable aleatoria \(X\), su función de distribución
\(F_X(x)\) y su función de densidad \(f_X(x)\). Considérese, además, un
suceso \(A \in \mathscr{F}\).

La \textbf{función de distribución de \(X\) condicionada por \(A\)} es:

\[F_X(x\, /\, A) = P(X \leq x \, /\, A) = \frac{P(\{X \leq x \bigcap A )\} }{P(A)}\]

Y la \textbf{función de densidad de \(X\) condicionada por \(A\)} es:

\[f_X(x \, /\, A)=\frac{dF_X(x \, /\, A)}{dx}=\lim \limits_{\Delta x \to 0} \frac{F_X(x+\Delta x/2 \, /\, A)-F_X(x-\Delta x/2 \, /\, A)}{\Delta x}= \\
=\lim \limits_{\Delta x \to 0} \frac{P\left(\{(x-\Delta x/2) < x \leq (x+\Delta x/2)\} \, /\, A \right)}{\Delta x} \]

    En general el suceso condicionante \(A\) podemos definirlo con
intervalos de la variable aleatoria. Suponiendo que
\(A=\{x_1\leq X \leq x_2\}\):

\begin{align*}
F_X(x\, /\, \{x_1\leq X \leq x_2\}) & = \frac{P(\{X \leq x \bigcap \{x_1\leq X \leq x_2\ )\} }{P(\{x_1\leq X \leq x_2\})}=\\
 & = \begin{cases}
    0       & \quad x < x_1\\
    \frac{F_X(x)-F_X(x_1^-)}{F_X(x_2)-F_X(x_1^-)}  & \quad x_1 \leq x \leq x_2\\
    1 &\quad x > x_2
  \end{cases}
\end{align*}

\begin{align*}
f_X(x\, /\, \{x_1\leq X \leq x_2\}) = \begin{cases}
    0       & \quad x < x_1\\
    \frac{f_X(x)}{F_X(x_2)-F_X(x_1^-)}  & \quad x_1 \leq x \leq x_2\\
    0 &\quad x > x_2
  \end{cases}
\end{align*}

    \hypertarget{teorema-de-bayes}{%
\subsubsection*{Teorema de Bayes}\label{teorema-de-bayes}}

Recordemos el \textbf{Teorema de Bayes} con dos sucesos:

\[P(A/B) = \frac{P(B/A)P(A)}{P(B)}\]

Si \(B=\{X \leq x\}\):

\[P(A\, / \, X \leq x) = \frac{F_X(x \, / \, A)P(A)}{F_X(x)} \iff F_X(x \, / \, A) = \frac{P(A\, / \, X \leq x)F_X(x)}{P(A)}\]

Si \(B=\{x_1 < X \leq x_2\}\):

\[P(A\, / \{x_1 < X \leq x_2\}) = \frac{F_X(x_2 \, / \, A)-F_X(x_1 \, / \, A)}{F_X(x_2)-F_X(x_1)}P(A)\]

    Si \(B=\{X=x\}\) y la variable aleatoria es continua nos encontramos con
un problema para condicionar por \(B\) pues su probabilidad es nula. En
tal caso consideramos que \(B=\{x<X-\Delta x/2 \leq x+\Delta x/2\}\) y
tomamos un paso al límite:

\begin{align*}
P(A \, / \, X=x) & = \lim \limits_{\Delta x \to 0} P(A\, / \{x-\Delta x/2 < X \leq x + \Delta x/2\}) \\
& = \lim \limits_{\Delta x \to 0} \frac{F_X(x +\Delta x/2 \, / \, A)-F_X(x-\Delta x/2 \, / \, A)}{F_X(x+\Delta x/2)-F_X(x-\Delta x/2)}P(A)
\end{align*}

Por tanto,

\[ P(A \, / \, X=x) = \frac{f_X(x \, / \, A)P(A)}{f_X(x)} \iff f_X(x \, / \, A) = \frac{P(A \, / \, X=x)f_X(x)}{P(A)}\]

    \hypertarget{teorema-de-la-probabilidad-total}{%
\subsubsection*{Teorema de la Probabilidad
Total}\label{teorema-de-la-probabilidad-total}}

Sea \(A_i \in \mathscr{P}(\Omega) \, i=1\ldots K\) es una partición del
espacio muestral \(\Omega\). Se obtiene inmediatamente el
\textbf{Teorema de la Probabilidad Total}:

\[F_X(x) = F_X(x \, / \, A_1)P(A_1)+\ldots + F_X(x \, / \, A_K)P(A_K)\]

Derivando:

\[f_X(x) = f_X(x \, / \, A_1)P(A_1)+\ldots + f_X(x \, / \, A_K)P(A_K)\]

Podemos obtener una versión continua del teorema integrando
\(f_X(x \, / \, A) = \frac{P(A \, / \, X=x)f_X(x)}{P(A)}\):

\[P(A) = \int_{-\infty}^{+\infty} P(A \, / \, X=x)f_X(x) dx\]

    \hypertarget{esperanza-y-momentos-condicionados}{%
\subsubsection*{Esperanza y momentos
condicionados}\label{esperanza-y-momentos-condicionados}}

\[\eta_{X/A} = E\{X \, / \, A\}=\int_{-\infty}^{\infty}xf_X(x\, / \, A)dx\]


    % Add a bibliography block to the postdoc
    
    
    
    \end{document}
